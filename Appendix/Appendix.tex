\chapter{Holographic Overlay System Details} \label{App:HolographicOverlaySystemDetails}
% Add a paragraph to introduce this

% The DICOM image 
The \gls{dicom} data generated by the \gls{ct} scanner was displayed using a real-time rendered iso-surface to represent 3D models by compiling the \gls{dicom} data from real-time using marching cubes~\cite{Lorensen} from \gls{dicom} information generated by fo-DICOM~\footnote{https://github.com/fo-dicom/fo-dicom}.
Running marching cubes, even in parallel blocks, was deemed too slow and imprecise.
This was then encased within an avatar of the patient for reviewing processes or could be viewed.

% The placing of electrodes
Later iterations of this project aimed at helping to communicate tasks for physiotherapists by communicating with them where electrodes should be placed on the patient. 
Predefined locations were based on previous data to simulate the concept of predefined instructions and the system's body type using the later work of Booij et al.~\cite {Booij2022}, where a more accurate model of the body could be formed to counteract issues like the patient's breathing.
This allowed us to correct the difference between different body types in different patients. 

% The final system
The final part of this system looked at making the \gls{ct} \gls{dicom} data more interactive. 
A two-way clipping plane that would show the current slice with adjustable contrast was included.
These six clipping planes were placed on the scan's vertical, horizontal, and depth axes.
These would each have their own data set that could be displayed, communicating what slice was being looked at in either slice/pixel range, percentage through the volume, or center meters through the volume. 
This system also came with several different methods of interacting with it, depending on the type of task required by the radiologist.
The final version of the product can be seen in \autoref{fig:HolographicOverlaySystem}. 

% The Direct Volume Rendering Section
This system was then extended to the present version of this project and moved over to having a \gls{dvr} system attached. 
\gls{dvr} works by directly rendering the data from the \gls{dicom} files. This allows for far greater data flexibility, rendering the visualization with almost no performance overheads. 
The trade-off for using \gls{dvr} is that it requires a much higher level of performance overall, requiring us to use a wireless theater using holographic remoting to a nearby \gls{pc} to run the system. 
This was an acceptable trade-off as this system to have both performed more reliably, delivered a smoother frame rate, improved the quality of graphics, and allowed for much faster network connectivity. 

% add a paragraph to end this

\chapter{X-ray Vision Literature Review Methodology} \label{app:XRayLitReviewMethodology}
% This literature review was inspired by the PRIMA 2020 SRC (Page et al. 2021) protocols checklist. It utilized five databases: 
% ACM Digital Library~\footnote{\url{https://dl.acm.org/}}, 
% IEEE Xplore~\footnote{\url{https://ieeexplore.ieee.org/Xplore/home.jsp}}, 
% Pub Med~\footnote{\url{https://pubmed.ncbi.nlm.nih.gov/}}, 
% Web of Science~\footnote{\url{https://www.webofscience.com/wos/}}, 
% and Scopus~\footnote{\url{https://www.scopus.com/}} 
% using the search terms:
% \textit{“("X-ray vision" OR "X-ray visualization" OR ("occlusion" AND "perception") OR (visualization AND x-ray) OR "see-through vision" OR "ghosted views" OR "augmented reality x-ray") AND ("AR" OR "Augmented Reality" OR "Mixed Reality")”}
% These terms were filtered by the paper’s Abstract, Title, and Keywords. To find grey literature, papers cited by more than five papers were also considered for entry into this database (2 papers found and accepted). For a paper to be accepted, it needed to meet the eligibility criteria outlined below. The process and results of this review as depicted in \autoref{fig:X-LItReviewMethodology}

% \begin{figure}
%     \centering
%     \includegraphics[width=\textwidth]{Chapter2/Images/LItReviewMethodology.pdf}
%     \caption{The protocol utilized for this literature review with the matching results. }
%     \label{fig:X-LItReviewMethodology}
% \end{figure}

% \subsubsection{Eligibility Criteria}
% All papers in this review needed to meet the following criteria:
% \begin{itemize}
%     \item X-ray vision (XRV) needed to be explored in the research, placing virtual objects behind real-world objects.
%     \item Mixed Reality Technologies other than VR must have been used.
%     \item Papers that simply superimposed data over the real world and papers that did not focus on making the content appear on the other side of or within the real-world object were not included.
%     \item Studies included in multiple papers were only recorded once all papers that highlighted the same research had been included in this review, as the content between both articles provides different amounts of information.
%     \item If the focus on XRV was light compared to other study elements and two or more reviewers agreed on it, then the work would not be included in favor of more focused research. 
% \end{itemize}

% These criteria ensured that the X-ray visualization was more than superimposing a medical image onto a human subject. Any paper that did not utilize \gls{ar} was removed. This meant that some documents that utilized Virtual Reality technologies to test \gls{ar} in a completely virtual environment were not included.

\chapter{Class Diagram for Random Generating Volumes} \label{app:RandomGerneratingVolumesClassDiagram}

\begin{figure}
    \centering
    \includegraphics[width=\textwidth]{Chapter4/Images/DefaultClassDiagram.png}
    \caption{The Class Diagram showing the main dependencies of the Random Generating Volumes System.}
    \label{fig:RandomGerneratingVolumesClassDiagram}
\end{figure}


\chapter{Chapter 5: Counting Everything Post Hoc}  \label{apendix:ErrorWhenCountingEverything}

% latex table generated in R 4.2.3 by xtable 1.8-4 package
% Fri May 12 14:44:16 2023

\begin{table}[ht]
\footnotesize
\centering
\begin{tabular}{llllrrr}
  \hline
  
LHS VIRT &  \begin{tabular}{@{}c@{}}LHS Amount\\Of Objects\end{tabular} & RHS VIRT & \begin{tabular}{@{}c@{}}RHS Amount\\Of Objects\end{tabular}  & estimate& t.ratio & p.value \\ 
  \hline
Halo/Outline  &  14  &  Hatching  &  14 & -2.76573 & -8.68107 & 0.00000 \\ 
  Halo/Outline  &  14  &  No Effect  &  14 & -1.04346 & -3.28904 & 0.11078 \\ 
  Halo/Outline  &  14  &  Stippling  &  14 & -2.10695 & -6.64124 & 0.00000 \\ 
  Halo/Outline  &  14  &  Halo/Outline  &  16 & -0.04346 & -0.13697 & 1.00000 \\ 
  Halo/Outline  &  14  &  Hatching  &  16 & -3.13869 & -9.89337 & 0.00000 \\ 
  Halo/Outline  &  14  &  No Effect  &  16 & -1.23393 & -3.88944 & 0.01542 \\ 
  Halo/Outline  &  14  &  Stippling  &  16 & -2.18631 & -6.89140 & 0.00000 \\ 
  Halo/Outline  &  14  &  Halo/Outline  &  18 & -0.39266 & -1.23770 & 0.99960 \\ 
  Halo/Outline  &  14  &  Hatching  &  18 & -3.93234 & -12.39501 & 0.00000 \\ 
  Halo/Outline  &  14  &  No Effect  &  18 & -1.37679 & -4.33973 & 0.00254 \\ 
  Halo/Outline  &  14  &  Stippling  &  18 & -2.78712 & -8.97394 & 0.00000 \\ 
  Halo/Outline  &  14  &  Halo/Outline  &  20 & -0.55139 & -1.73802 & 0.97364 \\ 
  Halo/Outline  &  14  &  Hatching  &  20 & -5.02791 & -15.96877 & 0.00000 \\ 
  Halo/Outline  &  14  &  No Effect  &  20 & -1.82812 & -5.78551 & 0.00000 \\ 
  Halo/Outline  &  14  &  Stippling  &  20 & -3.34638 & -10.50360 & 0.00000 \\ 
  Halo/Outline  &  14  &  Halo/Outline  &  22 & -0.61488 & -1.93816 & 0.92538 \\ 
  Halo/Outline  &  14  &  Hatching  &  22 & -5.20219 & -16.39764 & 0.00000 \\ 
  Halo/Outline  &  14  &  No Effect  &  22 & -2.36092 & -7.44176 & 0.00000 \\ 
  Halo/Outline  &  14  &  Stippling  &  22 & -4.10695 & -12.94537 & 0.00000 \\
  Hatching  &  14  &  No Effect  &  14 & 1.72228 & 5.38580 & 0.00002 \\ 
  Hatching  &  14  &  Stippling  &  14 & 0.65878 & 2.06011 & 0.87628 \\ 
  Hatching  &  14  &  Halo/Outline  &  16 & 2.72228 & 8.51293 & 0.00000 \\ 
  Hatching  &  14  &  Hatching  &  16 & -0.37296 & -1.16630 & 0.99983 \\ 
  Hatching  &  14  &  No Effect  &  16 & 1.53180 & 4.79015 & 0.00033 \\ 
  Hatching  &  14  &  Stippling  &  16 & 0.57942 & 1.81192 & 0.95998 \\ 
  Hatching  &  14  &  Halo/Outline  &  18 & 2.37307 & 7.42092 & 0.00000 \\ 
  Hatching  &  14  &  Hatching  &  18 & -1.16661 & -3.64816 & 0.03631 \\ 
  Hatching  &  14  &  No Effect  &  18 & 1.38894 & 4.34342 & 0.00250 \\ 
  Hatching  &  14  &  Stippling  &  18 & -0.02139 & -0.06822 & 1.00000 \\ 
  Hatching  &  14  &  Halo/Outline  &  20 & 2.21434 & 6.92455 & 0.00000 \\ 
  Hatching  &  14  &  Hatching  &  20 & -2.26217 & -7.12720 & 0.00000 \\ 
  Hatching  &  14  &  No Effect  &  20 & 0.93761 & 2.94296 & 0.26525 \\ 
  Hatching  &  14  &  Stippling  &  20 & -0.58065 & -1.80864 & 0.96068 \\ 
  Hatching  &  14  &  Halo/Outline  &  22 & 2.15085 & 6.72600 & 0.00000 \\ 
  Hatching  &  14  &  Hatching  &  22 & -2.43645 & -7.61913 & 0.00000 \\  
       \hline
\end{tabular}
\end{table}

\begin{table}[ht]
\footnotesize
\centering
\begin{tabular}{llllrrr}
  \hline
LHS VIRT & \begin{tabular}{@{}c@{}}RHS Amount\\Of Objects\end{tabular}  & RHS VIRT & \begin{tabular}{@{}c@{}}RHS Amount\\Of Objects\end{tabular}  & estimate & t.ratio & p.value \\ 
  \hline 
  Hatching  &  14  &  No Effect  &  22 & 0.40482 & 1.26592 & 0.99946 \\ 
  Hatching  &  14  &  Stippling  &  22 & -1.34122 & -4.19417 & 0.00468 \\
  No Effect  &  14  &  Stippling  &  14 & -1.06349 & -3.33926 & 0.09595 \\ 
  No Effect  &  14  &  Halo/Outline  &  16 & 1.00000 & 3.13990 & 0.16561 \\ 
  No Effect  &  14  &  Hatching  &  16 & -2.09524 & -6.57885 & 0.00000 \\ 
  No Effect  &  14  &  No Effect  &  16 & -0.19048 & -0.59808 & 1.00000 \\ 
  No Effect  &  14  &  Stippling  &  16 & -1.14286 & -3.58846 & 0.04431 \\ 
  No Effect  &  14  &  Halo/Outline  &  18 & 0.65079 & 2.04343 & 0.88396 \\ 
  No Effect  &  14  &  Hatching  &  18 & -2.88889 & -9.07083 & 0.00000 \\ 
  No Effect  &  14  &  No Effect  &  18 & -0.33333 & -1.04663 & 0.99997 \\ 
  No Effect  &  14  &  Stippling  &  18 & -1.74367 & -5.58808 & 0.00001 \\ 
  No Effect  &  14  &  Halo/Outline  &  20 & 0.49206 & 1.54503 & 0.99286 \\ 
  No Effect  &  14  &  Hatching  &  20 & -3.98445 & -12.60529 & 0.00000 \\ 
  No Effect  &  14  &  No Effect  &  20 & -0.78467 & -2.47333 & 0.60495 \\ 
  No Effect  &  14  &  Stippling  &  20 & -2.30292 & -7.20155 & 0.00000 \\ 
  No Effect  &  14  &  Halo/Outline  &  22 & 0.42857 & 1.34567 & 0.99875 \\ 
  No Effect  &  14  &  Hatching  &  22 & -4.15873 & -13.05801 & 0.00000 \\ 
  No Effect  &  14  &  No Effect  &  22 & -1.31746 & -4.13670 & 0.00592 \\ 
  No Effect  &  14  &  Stippling  &  22 & -3.06349 & -9.61907 & 0.00000 \\ 
  Stippling  &  14  &  Halo/Outline  &  16 & 2.06349 & 6.47917 & 0.00000 \\ 
  Stippling  &  14  &  Hatching  &  16 & -1.03175 & -3.23958 & 0.12711 \\ 
  Stippling  &  14  &  No Effect  &  16 & 0.87302 & 2.74119 & 0.39865 \\ 
  Stippling  &  14  &  Stippling  &  16 & -0.07937 & -0.24920 & 1.00000 \\ 
  Stippling  &  14  &  Halo/Outline  &  18 & 1.71429 & 5.38269 & 0.00002 \\ 
  Stippling  &  14  &  Hatching  &  18 & -1.82540 & -5.73157 & 0.00000 \\ 
  Stippling  &  14  &  No Effect  &  18 & 0.73016 & 2.29263 & 0.73955 \\ 
  Stippling  &  14  &  Stippling  &  18 & -0.68018 & -2.17982 & 0.81246 \\ 
  Stippling  &  14  &  Halo/Outline  &  20 & 1.55556 & 4.88429 & 0.00021 \\ 
  Stippling  &  14  &  Hatching  &  20 & -2.92096 & -9.24080 & 0.00000 \\ 
  Stippling  &  14  &  No Effect  &  20 & 0.27882 & 0.87887 & 1.00000 \\ 
  Stippling  &  14  &  Stippling  &  20 & -1.23943 & -3.87587 & 0.01622 \\ 
  Stippling  &  14  &  Halo/Outline  &  22 & 1.49206 & 4.68494 & 0.00054 \\ 
  Stippling  &  14  &  Hatching  &  22 & -3.09524 & -9.71875 & 0.00000 \\ 
  Stippling  &  14  &  No Effect  &  22 & -0.25397 & -0.79744 & 1.00000 \\ 
  Stippling  &  14  &  Stippling  &  22 & -2.00000 & -6.27981 & 0.00000 \\ 
  Halo/Outline  &  16  &  Hatching  &  16 & -3.09524 & -9.71875 & 0.00000 \\ 
  Halo/Outline  &  16  &  No Effect  &  16 & -1.19048 & -3.73798 & 0.02665 \\ 
  Halo/Outline  &  16  &  Stippling  &  16 & -2.14286 & -6.72837 & 0.00000 \\ 
  Halo/Outline  &  16  &  Halo/Outline  &  18 & -0.34921 & -1.09647 & 0.99993 \\ 
  Halo/Outline  &  16  &  Hatching  &  18 & -3.88889 & -12.21074 & 0.00000 \\ 
  Halo/Outline  &  16  &  No Effect  &  18 & -1.33333 & -4.18654 & 0.00483 \\ 
  Halo/Outline  &  16  &  Stippling  &  18 & -2.74367 & -8.79286 & 0.00000 \\ 
  Halo/Outline  &  16  &  Halo/Outline  &  20 & -0.50794 & -1.59487 & 0.98967 \\ 
  Halo/Outline  &  16  &  Hatching  &  20 & -4.98445 & -15.76891 & 0.00000 \\ 
  Halo/Outline  &  16  &  No Effect  &  20 & -1.78467 & -5.62540 & 0.00000 \\ 
  Halo/Outline  &  16  &  Stippling  &  20 & -3.30292 & -10.32869 & 0.00000 \\ 
  Halo/Outline  &  16  &  Halo/Outline  &  22 & -0.57143 & -1.79423 & 0.96365 \\ 
  Halo/Outline  &  16  &  Hatching  &  22 & -5.15873 & -16.19792 & 0.00000 \\ 
  Halo/Outline  &  16  &  No Effect  &  22 & -2.31746 & -7.27660 & 0.00000 \\ 
  Halo/Outline  &  16  &  Stippling  &  22 & -4.06349 & -12.75897 & 0.00000 \\ 
  Hatching  &  16  &  No Effect  &  16 & 1.90476 & 5.98077 & 0.00000 \\ 
  Hatching  &  16  &  Stippling  &  16 & 0.95238 & 2.99038 & 0.23834 \\ 
\hline
\end{tabular}
\end{table}


\begin{table}[ht]
\footnotesize
\centering
\begin{tabular}{llllrrr}
\hline
LHS VIRT & \begin{tabular}{@{}c@{}}LHS Amount\\Of Objects\end{tabular} & RHS VIRT & \begin{tabular}{@{}c@{}}RHS Amount\\Of Objects\end{tabular} & estimate & t.ratio & p.value \\ 
  \hline 
  Hatching  &  16  &  Halo/Outline  &  18 & 2.74603 & 8.62228 & 0.00000 \\ 
  Hatching  &  16  &  Hatching  &  18 & -0.79365 & -2.49199 & 0.59037 \\ 
  Hatching  &  16  &  No Effect  &  18 & 1.76190 & 5.53221 & 0.00001 \\ 
  Hatching  &  16  &  Stippling  &  18 & 0.35157 & 1.12671 & 0.99990 \\ 
  Hatching  &  16  &  Halo/Outline  &  20 & 2.58730 & 8.12388 & 0.00000 \\ 
  Hatching  &  16  &  Hatching  &  20 & -1.88921 & -5.97675 & 0.00000 \\ 
  Hatching  &  16  &  No Effect  &  20 & 1.31057 & 4.13100 & 0.00605 \\ 
  Hatching  &  16  &  Stippling  &  20 & -0.20768 & -0.64946 & 1.00000 \\ 
  Hatching  &  16  &  Halo/Outline  &  22 & 2.52381 & 7.92452 & 0.00000 \\ 
  Hatching  &  16  &  Hatching  &  22 & -2.06349 & -6.47917 & 0.00000 \\ 
  Hatching  &  16  &  No Effect  &  22 & 0.77778 & 2.44215 & 0.62916 \\ 
  Hatching  &  16  &  Stippling  &  22 & -0.96825 & -3.04022 & 0.21205 \\ 
  No Effect  &  16  &  Stippling  &  16 & -0.95238 & -2.99038 & 0.23834 \\ 
  No Effect  &  16  &  Halo/Outline  &  18 & 0.84127 & 2.64151 & 0.47344 \\ 
  No Effect  &  16  &  Hatching  &  18 & -2.69841 & -8.47276 & 0.00000 \\ 
  No Effect  &  16  &  No Effect  &  18 & -0.14286 & -0.44856 & 1.00000 \\ 
  No Effect  &  16  &  Stippling  &  18 & -1.55319 & -4.97764 & 0.00013 \\ 
  No Effect  &  16  &  Halo/Outline  &  20 & 0.68254 & 2.14311 & 0.83363 \\ 
  No Effect  &  16  &  Hatching  &  20 & -3.79397 & -12.00269 & 0.00000 \\ 
  No Effect  &  16  &  No Effect  &  20 & -0.59419 & -1.87294 & 0.94517 \\ 
  No Effect  &  16  &  Stippling  &  20 & -2.11245 & -6.60591 & 0.00000 \\ 
  No Effect  &  16  &  Halo/Outline  &  22 & 0.61905 & 1.94375 & 0.92347 \\ 
  No Effect  &  16  &  Hatching  &  22 & -3.96825 & -12.45994 & 0.00000 \\ 
  No Effect  &  16  &  No Effect  &  22 & -1.12698 & -3.53862 & 0.05211 \\ 
  No Effect  &  16  &  Stippling  &  22 & -2.87302 & -9.02099 & 0.00000 \\ 
  Stippling  &  16  &  Halo/Outline  &  18 & 1.79365 & 5.63189 & 0.00000 \\ 
  Stippling  &  16  &  Hatching  &  18 & -1.74603 & -5.48237 & 0.00001 \\ 
  Stippling  &  16  &  No Effect  &  18 & 0.80952 & 2.54183 & 0.55123 \\ 
  Stippling  &  16  &  Stippling  &  18 & -0.60081 & -1.92547 & 0.92957 \\ 
  Stippling  &  16  &  Halo/Outline  &  20 & 1.63492 & 5.13349 & 0.00006 \\ 
  Stippling  &  16  &  Hatching  &  20 & -2.84159 & -8.98972 & 0.00000 \\ 
  Stippling  &  16  &  No Effect  &  20 & 0.35819 & 1.12903 & 0.99990 \\ 
  Stippling  &  16  &  Stippling  &  20 & -1.16006 & -3.62768 & 0.03890 \\ 
  Stippling  &  16  &  Halo/Outline  &  22 & 1.57143 & 4.93413 & 0.00016 \\ 
  Stippling  &  16  &  Hatching  &  22 & -3.01587 & -9.46955 & 0.00000 \\ 
  Stippling  &  16  &  No Effect  &  22 & -0.17460 & -0.54824 & 1.00000 \\ 
  Stippling  &  16  &  Stippling  &  22 & -1.92063 & -6.03061 & 0.00000 \\ 
  Halo/Outline  &  18  &  Hatching  &  18 & -3.53968 & -11.11426 & 0.00000 \\ 
  Halo/Outline  &  18  &  No Effect  &  18 & -0.98413 & -3.09006 & 0.18781 \\ 
  Halo/Outline  &  18  &  Stippling  &  18 & -2.39446 & -7.67373 & 0.00000 \\ 
  Halo/Outline  &  18  &  Halo/Outline  &  20 & -0.15873 & -0.49840 & 1.00000 \\ 
  Halo/Outline  &  18  &  Hatching  &  20 & -4.63524 & -14.66415 & 0.00000 \\ 
  Halo/Outline  &  18  &  No Effect  &  20 & -1.43546 & -4.52468 & 0.00113 \\ 
  Halo/Outline  &  18  &  Stippling  &  20 & -2.95372 & -9.23667 & 0.00000 \\ 
  Halo/Outline  &  18  &  Halo/Outline  &  22 & -0.22222 & -0.69776 & 1.00000 \\ 
  Halo/Outline  &  18  &  Hatching  &  22 & -4.80952 & -15.10144 & 0.00000 \\ 
  Halo/Outline  &  18  &  No Effect  &  22 & -1.96825 & -6.18013 & 0.00000 \\ 
  Halo/Outline  &  18  &  Stippling  &  22 & -3.71429 & -11.66250 & 0.00000 \\ 
  Hatching  &  18  &  No Effect  &  18 & 2.55556 & 8.02420 & 0.00000 \\ 
  Hatching  &  18  &  Stippling  &  18 & 1.14522 & 3.67018 & 0.03369 \\ 
  Hatching  &  18  &  Halo/Outline  &  20 & 3.38095 & 10.61586 & 0.00000 \\ 
\hline
\end{tabular}
\end{table}

\begin{table}[ht]
\footnotesize
\centering
\begin{tabular}{llllrrr}
\hline
LHS VIRT & \begin{tabular}{@{}c@{}}LHS Amount\\Of Objects\end{tabular} & RHS VIRT & \begin{tabular}{@{}c@{}}RHS Amount\\Of Objects\end{tabular}  & estimate & t.ratio & p.value \\ 
\hline
  Hatching  &  18  &  Hatching  &  20 & -1.09556 & -3.46594 & 0.06557 \\ 
  Hatching  &  18  &  No Effect  &  20 & 2.10422 & 6.63264 & 0.00000 \\ 
  Hatching  &  18  &  Stippling  &  20 & 0.58597 & 1.83240 & 0.95539 \\ 
  Hatching  &  18  &  Halo/Outline  &  22 & 3.31746 & 10.41651 & 0.00000 \\ 
  Hatching  &  18  &  Hatching  &  22 & -1.26984 & -3.98718 & 0.01066 \\ 
  Hatching  &  18  &  No Effect  &  22 & 1.57143 & 4.93413 & 0.00016 \\ 
  Hatching  &  18  &  Stippling  &  22 & -0.17460 & -0.54824 & 1.00000 \\ 
  No Effect  &  18  &  Stippling  &  18 & -1.41033 & -4.51982 & 0.00116 \\ 
  No Effect  &  18  &  Halo/Outline  &  20 & 0.82540 & 2.59167 & 0.51213 \\ 
  No Effect  &  18  &  Hatching  &  20 & -3.65112 & -11.55075 & 0.00000 \\ 
  No Effect  &  18  &  No Effect  &  20 & -0.45134 & -1.42264 & 0.99741 \\ 
  No Effect  &  18  &  Stippling  &  20 & -1.96959 & -6.15917 & 0.00000 \\ 
  No Effect  &  18  &  Halo/Outline  &  22 & 0.76190 & 2.39231 & 0.66721 \\ 
  No Effect  &  18  &  Hatching  &  22 & -3.82540 & -12.01138 & 0.00000 \\ 
  No Effect  &  18  &  No Effect  &  22 & -0.98413 & -3.09006 & 0.18781 \\ 
  No Effect  &  18  &  Stippling  &  22 & -2.73016 & -8.57244 & 0.00000 \\ 
  Stippling  &  18  &  Halo/Outline  &  20 & 2.23573 & 7.16503 & 0.00000 \\ 
  Stippling  &  18  &  Hatching  &  20 & -2.24078 & -7.23731 & 0.00000 \\ 
  Stippling  &  18  &  No Effect  &  20 & 0.95900 & 3.08777 & 0.18888 \\ 
  Stippling  &  18  &  Stippling  &  20 & -0.55925 & -1.78354 & 0.96574 \\ 
  Stippling  &  18  &  Halo/Outline  &  22 & 2.17224 & 6.96155 & 0.00000 \\ 
  Stippling  &  18  &  Hatching  &  22 & -2.41506 & -7.73975 & 0.00000 \\ 
  Stippling  &  18  &  No Effect  &  22 & 0.42621 & 1.36590 & 0.99848 \\ 
  Stippling  &  18  &  Stippling  &  22 & -1.31982 & -4.22975 & 0.00404 \\ 
  Halo/Outline  &  20  &  Hatching  &  20 & -4.47651 & -14.16199 & 0.00000 \\ 
  Halo/Outline  &  20  &  No Effect  &  20 & -1.27673 & -4.02435 & 0.00923 \\ 
  Halo/Outline  &  20  &  Stippling  &  20 & -2.79499 & -8.74030 & 0.00000 \\ 
  Halo/Outline  &  20  &  Halo/Outline  &  22 & -0.06349 & -0.19936 & 1.00000 \\ 
  Halo/Outline  &  20  &  Hatching  &  22 & -4.65079 & -14.60304 & 0.00000 \\ 
  Halo/Outline  &  20  &  No Effect  &  22 & -1.80952 & -5.68173 & 0.00000 \\ 
  Halo/Outline  &  20  &  Stippling  &  22 & -3.55556 & -11.16410 & 0.00000 \\ 
  Hatching  &  20  &  No Effect  &  20 & 3.19978 & 10.16259 & 0.00000 \\ 
  Hatching  &  20  &  Stippling  &  20 & 1.68153 & 5.29782 & 0.00003 \\ 
  Hatching  &  20  &  Halo/Outline  &  22 & 4.41302 & 13.96112 & 0.00000 \\ 
  Hatching  &  20  &  Hatching  &  22 & -0.17428 & -0.55135 & 1.00000 \\ 
  Hatching  &  20  &  No Effect  &  22 & 2.66699 & 8.43734 & 0.00000 \\ 
  Hatching  &  20  &  Stippling  &  22 & 0.92096 & 2.91356 & 0.28284 \\ 
  No Effect  &  20  &  Stippling  &  20 & -1.51825 & -4.76549 & 0.00037 \\ 
  No Effect  &  20  &  Halo/Outline  &  22 & 1.21324 & 3.82422 & 0.01960 \\ 
  No Effect  &  20  &  Hatching  &  22 & -3.37406 & -10.63526 & 0.00000 \\ 
  No Effect  &  20  &  No Effect  &  22 & -0.53279 & -1.67939 & 0.98165 \\ 
  No Effect  &  20  &  Stippling  &  22 & -2.27882 & -7.18300 & 0.00000 \\ 
  Stippling  &  20  &  Halo/Outline  &  22 & 2.73149 & 8.54175 & 0.00000 \\ 
  Stippling  &  20  &  Hatching  &  22 & -1.85581 & -5.80337 & 0.00000 \\ 
  Stippling  &  20  &  No Effect  &  22 & 0.98546 & 3.08167 & 0.19175 \\ 
  Stippling  &  20  &  Stippling  &  22 & -0.76057 & -2.37841 & 0.67764 \\ 
  Halo/Outline  &  22  &  Hatching  &  22 & -4.58730 & -14.40369 & 0.00000 \\ 
  Halo/Outline  &  22  &  No Effect  &  22 & -1.74603 & -5.48237 & 0.00001 \\ 
  Halo/Outline  &  22  &  Stippling  &  22 & -3.49206 & -10.96474 & 0.00000 \\ 
  Hatching  &  22  &  No Effect  &  22 & 2.84127 & 8.92131 & 0.00000 \\ 
  Hatching  &  22  &  Stippling  &  22 & 1.09524 & 3.43894 & 0.07126 \\ 
  No Effect  &  22  &  Stippling  &  22 & -1.74603 & -5.48237 & 0.00001 \\ 
   \hline
\end{tabular}
\end{table}

\chapter{Chapter 3's Participant Comments on X-Ray Vision} \label{app:Chapter3Comments}
This appendix displays users' comments on the study in \autoref{Chap:X-ray Implemntion}. Results like space and N/A have been omitted from all of the sections in this appendix and have been anonymized. 

\section{Favorite Visualizations and Why}
\begin{itemize}
  \item Favorite: \textbf{Wire Frame}, Why: Consistent lines and i could see all objects clearly. 
  \item Favorite: \textbf{Random Dot}, Why: I could use it as a grid to assist in placing the shape.  
  \item Favorite: \textbf{None}, Why: Easy to see the pointer and object
  \item Favorite: \textbf{Saliency}, Why: It was easier to position the objects accurately as opposed to the others.
  \item Favorite: \textbf{None}, Why: They are pretty much all the same to me minus none does not have a bothersome outline  
  \item Favorite: \textbf{Wire Frame}, Why: It created a better frame of reference in my opinion
  \item Favorite: \textbf{Random Dot}, Why: Didn't freak out as much as edge based and saliency and  gave me a good idea of what I was looking at
  \item Favorite: \textbf{Wire Frame}, Why: Helped me to place the object with more confident.
  \item Favorite: \textbf{None}, Why: No obstacles to obstruct vision 
  \item Favorite: \textbf{Saliency}, Why: Because it was fun to try to place the object with certain sections blacked out
  \item Favorite: \textbf{Wire Frame}, Why: It helped to have corners I could use for reference. I thought it might be better though if the mesh was more of a grid.
  \item Favorite: \textbf{Wire Frame}, Why: I felt like it gave me a better  understanding  of the location of objects
  \item Favorite: \textbf{Random Dot}, Why: Gave a good impression of the outer edge of the box, the cubes matching the shape helped distinguish edges and the gave good reference points for placement
  \item Favorite: \textbf{None}, Why: The back plane in the none condition gave me enough cue and least cluttered view.
  \item Favorite: \textbf{Edge based}, Why: Gave me a reference to where I was pointing (I felt like I was being guided more). Other than that none provided the least amount of visual clutter. 
  \item Favorite: \textbf{Random Dot}, Why: Not too sure why. It did not occlude my view. I felt that it just easier to use than the others.
  \item Favorite: \textbf{Edge based}, Why: I feel that the wireframe and random dot conditions felt the most natural due to the uniform nature of the pattern designs. The edge-based and saliency patterns were a little bit too confusing visually and I felt that I couldn't judge the depth as well with these patterns for some reason.
  \item Favorite: \textbf{None}, Why: There was no distraction lines, etc. so it was much easier to place the object.
  \item Favorite: \textbf{Edge based}, Why: Edge base, as it has some points of references, but not enough to hide the position of the objects.
  \item Favorite: \textbf{Wire Frame}, Why: This one gave the best 3d visual reference to tell where I was placing the object compared to the others.
\end{itemize}

\section{General Comments}
\begin{enumerate}
  \item Very interesting.  I think having some form of guidance helps a lot compared to none. Saliency was very frustrating though as I couldn't see the objects from the views I wanted to.
  \item Saliency was the hardest one because my sight of the shapes were often blocked. 
  \item Everything's are interesting and satisfied. But saliency bit hard to recognise object
  \item Cool experiment! It would be good if the box position was higher around the waist or chest for ease of use.
  \item I personally don't fine myself using any of the vis as I think the box and a hexagon  or ngon is enough to judge the xyz positions  
  \item Edge based and saliency made the task harder than with no visualisation
  \item Very interesting experience, looking forward to more AR research!
  \item I found even the slight delay from moving the headset disorientating (especially when holding the orb thing)
  \item Wireframe or none are close, with none the back edge is clearer and the front of the box is like the wire frame 
  \item If there were no back plane in the none condition then I would have chosen the wire frame condition as my preference.
  \item I think a grid pattern could be interesting. 
  \item The tracking was cutting out when I was standing in front of the cube so I was limited to manipulating the object from one specific angle. The precision was not great using the fishing reel manipulation and was abit jittery.
  \item I was not always sure if I could see the back faces of the cubic workspace when working in every condition, but I think that the system could benefit from a brighter or easier-to-see backface edge design. This way I could relate the placement of objects relative to not only the faces positioned toward me, but relative to the Cuba's rear sides/edges as well.
  \item Controller issues, but that probably can be resolved at a later stage.
  \item I really enjoyed participating in the study and getting to use the hololens technology.
\end{enumerate}


\chapter{Feedback from Chapter 5 Examing the Effect of Perception on VIRTs} \label{app:Chapter5Comments}
This section of the appendix displays the comments users had on the study in \autoref{Chap:X-ray Implemntion}. Results like empty space and N/A have been omitted from all of the sections in this appendix and have been anonymized. 

\section{Things That Were Liked and Disliked About VIRTs}
\subsection{Halo}
\subsubsection{I liked this visualization because.}
\begin{itemize}
  \item The outline effect was much clearer than the no effect as it made counting the nodes easier. 
  \item The objects are easyly to see
  \item it have the outline to idicated the obh=ject which allows me to identiflied where is the object and how many so i won't miss the object now
  \item outline makes it very easy to count green shapes
  \item It was very easy to count the green objects because of the white outline.
  \item Objects towards the back of the frame were clearly highlighted, whereas in other conditions they were dim and sometimes impossible to see from the opposite side
  \item it outline the objects, i can just sit and see the objects easily without moving around 
  \item The halo effect added a lot of clarity to each object, which made it obvious an object existed even if it was very small. Combined with the inherant colour changes of the object made it quite clear to me.
  \item it's easy to use and because it's 3D interface, I can check in different angle
  \item I could easily idenity all shapes.
  \item The outline effect made it easier to identify and differentiate betwen the blue circles.
  \item Simple and effective. Easy to read, smooth.
  \item There was a good amount of clarity to identify the green things
  \item The shapes of the objects are very clear
  \item Very easy to count green objects (counting total condition)
  \item It made it easier to count and view objects that were overlapping with others
  \item outline makes things a lot clearer
  \item the outline made it way easier to spot the blobs
  \item it was easy to notice the shape of the object so I could quickly count the numbers.
  \item It's easy and clear to count
  \item with just outline it was much easier to count the green objects. 
  \item It is easy to locate all the green objects (in total). It is easy to seperate the green objects because there is a boundary around each of them. The appearnce of the material changes slightly with respect to the viewpoint.
  \item I needed to move my head less often
\end{itemize}
\subsubsection{I disliked this visualization because.}
\begin{itemize}
  \item Some of the objects are not easy to see
  \item it is still hard to tell if the onejct was inside other object or not , because of the colour and sometimes , other object block the sign of those object behind it includet he outline.
  \item Dont know
  \item no 
  \item Sometimes on very small objects the halo effect was much stronger than the base colour of the object, making it sometimes difficult to determine the object's colour. 
  \item Sometimes it was hard to identify the depth of a shape. The outline made them look a bit flat.
  \item Good visualisation, but it requires alot of concentration to keep track of the blue circles while counting. (especially as the number of blue circles in the visualisation is high).
  \item -
  \item It is hard to recognize whether the white dots in the back are in the boundary of blue object
  \item Little bit hard to find green objects when it placed behind blue cells.
  \item I didnt have any dislikes
  \item sometimes made me too confident so I had to recount
  \item none
  \item it was sometimes hard to detect when the object was too small.
  \item It is a little hard to see how deep is the green object inside the blue objects. The green object has to be far away from the blue object, in order to decide that it is definetly not inside. But when the green object is just infront of the blue object, I felt it is hard to say if it is inside it or outside it. I would prefered that the material of the object could tell more about its depth or 3D structure.
\end{itemize}
\subsection{Hatching}
\subsubsection{I liked this visualization because.}
\begin{itemize}
  \item Some of the objects are easy to see
  \item It have a web to identiflied the border of the whole red object so I won't missunderstand the background outside of the object as the green object
  \item easy to catch the edges of shapes
  \item It highlight the boundary of the objects, that help me to find out the objects, but not helping much. 
  \item The alternating cross-hatching density (direction?) occasionally made it clear where there were two separate objects. 
  \item It is visually the most unique.
  \item It was still kind of useful for identifying the green objects I think but it would be hard for me to use long-term.
  \item The shape of blue object is very easy to recognize
  \item This condition is easy to count total green objects.
  \item I didn't have any likes for this visulization
  \item Sometimes when overlapping crosshatching helped
  \item it helps to notice depth information at some degree.
  \item The opacity of the color of the object seems to change as a function of distance, which helped in identifying the objects. The strides (or lines) on the object assisted in separating the objects (I could tell the nearby objects were disconncted rather than perceiving them as a one big object)
\end{itemize}
\subsubsection{I disliked this visualization because.}
\begin{itemize}
  \item This visualization was least preferred as the nodes were not easy to count at all. It was straineous to count them. 
  \item Some objects are difficults
  \item the web strongly interupt my adjustment on green obejct's position adn existance.
  \item still difficult to make out distant shapes, slightly worse performance compared to no effect
  \item the billboard view of the objects, which make me difficult to see the side of the red objects and count the object number. Also, the cross-line could be slightly transparent, and the inside object cross-line could be more obvious. 
  \item The Hatching was very heavy and often felt like it was occluding the already sometimes faint colours of distant/deep objects. The effect helped to show the blue objects but did not help represent the green ones. Additionally, the effect felt relative to my view rather than the world, meaning looking around the visualisation was not very natural and a bit confusing. 
  \item it's hard to count the number because the cross hatching make me confuse
  \item I really struggled to tell anything about and when I moved my head, because the visual changed, I couldn't be certain that I hadn't already counted a shape. I had no real idea what any shapes actually looked like. If the grid didn't move with my head, maybe it would have been easier. This was the only visualization that gave me eye strain.
  \item The grid patterns was starting to give me a headache. I had to try to keep my head still to avoid physical discomfort.
  \item To my eyes it seemed glitchy (but maybe that's how it's supposed to be). Slightly laggy unfrotunately but overall really really hard to use.
  \item The objects were much harder to identify due to their shapes and their lack of clarity
  \item It is really lagging and makes me feel sick, the green object in the depth faded in the blue object. 
  \item Difficult to distinguish when green objects are in blue cells or placing close to other objects
  \item The visualisation made the objects seem very custered and made me confused during counting objects that were within close proximity to each other
  \item Crosshatching the big red blob itself made it way too cluttered. I think if only the green and blue were I would have had an easier time and maybe even rated it highly. But with the red one crosshatched there is just too much going on
  \item Hatching made it difficult to see into objects
  \item the lines made it distracting to see the blobs, i had difficulties in determining for both within blue blob and whole volume
  \item I had to rotate my view to find out whether the green objects were overlapped or not.
  \item it's not a little clear to see
  \item Some of the overlapping blue objects were hard to figure out.
  \item The opacity of the objects in the furthest point (too far away) was fading that I couldn't decisively determine if it was an object or not. I wish I could rotate the scene to look behind the big blue objects.
  \item It seemed to me that I had too much information
\end{itemize}
\subsection{No VIRT}
\subsubsection{I liked this visualization because.}
\begin{itemize}
  \item It was clear and easy to differentiate the colors of the artifacts
  \item it is simple enough for everyone to use it,
  \item It's simple and clean. Transparencies are a good way to perceive depth
  \item It was easy to figure out which green objects were inisde the blue shapes and which weren't
  \item Straightforward, interesting colours
  \item 3d view of the objects
  \item Without additional effects the visualisation was very clear to view. I think this made aspects of the visualisation more comfortable to parse.
  \item It was easy to see green shapes through green shapes (as in they looked off color and it help identify some hiding behind others).
  \item Everything felt clearer on the screen, so think I performed better at keeping track when counting green objects. Because of this I felt like the task was easier, despite depth probably being harder to interpret. Additionally, I probably took my time abit longer and used more head movement to carefully interpret the data with this one compared to the grid visualisation (because the grid was causing physical discomfort when I moved too much).
  \item Simple, smooth
  \item It was easier to identify each object individually, there was nothing reducing clarity
  \item The green objects are obviously in the blue object
  \item Low complexity and relatively easy to learn.
  \item Didnt have any particular likes for the visualizations
  \item No clutter, generally easy to see.
  \item There were less distractions which made it easier for me to spot the green blobs for both within the blue volumes and whole volume
  \item somehow seems natural for me.
  \item In some cases it was easier to identify number of green objects due to color shades.
  \item It is relatively easy to identidy the little green objects (in total) as the space is rather empty of everything but colors. The opacity or color seems to change relative to the depth and compostion of the object.
\end{itemize}
\subsubsection{I disliked this visualization because.}
\begin{itemize}
  \item Most of the objects are difficult to see and count
  \item it is hard to identified which one is green, and sometimes blue objects were blocking the green one, causing me need more time and attention to identify, sometimes I even missed it 
  \item Don't know
  \item I couldn't see the green objects far in the back
  \item Colours can bleed together, it can be difficult to determine blue pockets towards the back of the volume
  \item Sometimes, when the objects are far behind the objects, they can't really see them clearly. 
  \item Very small or far back green objects became difficult to see without significant head movement as their coloring blended a lot with the surrounding volume. Also, looking through green objects produced an "xray" effect, which was inconsistent with the external visualisation. 
  \item This visualization do not have an outline. Sometimes it's hard to count the number, and the 3D object is not too clear.
  \item I really struggled with being confident that I was identifying all the shapes near the back. Also some shapes would kind of blend together and I'd have to move my head a lot to see them clearer.
  \item I felt that the depth was probably harder to interpret compared to some of the other visualizations (specifically the visualization where objects were outlined \& also the staple one).
  \item A bit hard to count with it. Sometimes, green thingies hide behind other green thingies, and this visualization does not help in realizing that.
  \item THe green objects in the depth are hardly recognized
  \item It is Difficult to percept depth, so I couldn't recognize whether green objects are in the blue or not.
  \item It felt like i had to strain my eyes too much in order to try and correclty count objects as it was hard to determine where one object started and another ended if they were overlapping
  \item Sometimes confused if a green blob is half in a blue one or fully in one
  \item after a while, it did made me a bit dizzy looking at it
  \item It was difficult to figure out whether the green object is inside or behind the blue object.
  \item some of the opposite sided objects are hard to identify if they are single or double (one over other)
  \item It is harder to identify the relation between the blue and the green objects. In other words, it is hard to say if the green object is inside or outside the blue object.
  \item I might have missed some objects due to occlusion and thickness of red and blue 'fluids'
\end{itemize}
\subsection{Stippling}
\subsubsection{I liked this visualization because.}
\begin{itemize}
  \item Almost all the objects are easy to see and count
  \item because it has the dot to identify the size of the object so I tell it's position once I spotted it 
  \item LIttle dots help see green elelemts
  \item easier to spot the green objects because of the white spots on them
  \item More legible than the others so far
  \item The dots on the objects helped me to differenciate different objects 
  \item The Stippling seemed to serve to add texture to the objects, which helped with perceiving object depth, and also identifying the shape of objects, which was important due to some green blobs overlapping in various ways. 
  \item it's clear
  \item It was easy on the eyes. Seemed sort of magical.
  \item I feel like the task would probably be harder without the visualisation because it overall made the green objects easier to identify.
  \item I felt more confident answering. It feels like the green thingies are more visible.
  \item With the dots on the objects, it is easy to recognize the shapes of them. 
  \item It has low mental load and low complexity.
  \item It was easy to notice objects that were behind others
  \item Sometimes the white dots help difference between overlapping ones
  \item the green blobs were easy to count
  \item glitering surfaces were helpful for depth perception.
  \item It is easy to locate the objects in total. The material is helping alot I felt that the reflectance changes based on the viewpoint and the dotts on the surface of the object are helping as well. It is easy to differentiate between the green objects (even when many green objects are too close, I can tell they are not one object but three that are too close). It is easy to tell whether the green object is inside or outside the blue object.
\end{itemize}
\subsubsection{I disliked this visualization because.}
\begin{itemize}
  \item It was not clear as compared to the outline effect. At times, I doubted if I had seen the node properly. 
  \item it is still confusing about the green object is within or not of the blue object because sometime the dot mixed up together and looks like half is within and half is outside
  \item When having several shapes it gets confusing.
  \item Slight lag, but forgiveable
  \item sometimes. the objects are way behind other objects, then i need to move around to see it. 
  \item The effect was very noisy and was uncomfortable at first due to the amount of detail in the scene. 
  \item I found it hard to identify which dots where for each shape.
  \item The main issue was that sometimes when green objects were close together it was difficult for me to identify whether it was a single or multiple objects \& as a result had to make a guess sometimes since it was unclear to me.
  \item Slightly laggy
  \item Like the cross-hatching, it lowered the clarity and made it harder to identify between objects
  \item The white dots sometimes confuse me about the boundaries. 
  \item Difficult to distinguish when objects are grouped close.
  \item Felt clustered
  \item Still wasn't too clear overall sometimes, though. can't describe much but from training I'm guessing outline is going to be my fav
  \item I wasn't really sure if the number I put in is right because I'm counting all the blobs that touch the blue space
  \item sometimes it distracts me for counting the number.
  \item due to too much noise (dots), it was creating confusion.
  \item My depth perception of point clouds was not as good as with NOTHING qand OUTLINE conditions
\end{itemize}

\chapter{Feedback from Chapter 6 Examining the limits of Depth Perception} \label{app:Chapter6Comments}
This Appendix focuses of the written feedback that was collected during the study form \autoref{chap:DepthPerception}. Results like empty space and N/A have been omitted from all of the sections in this appendix and have been anonymized. 

\section{Things That Were Liked and Disliked About VIRTs}
This section contains a list of all of the user comments related to things participants liked and disliked about the \glspl{virt} can be found here.

\subsection{Halo}
\subsubsection{I liked this visualization because.}
\begin{itemize}
  \item some objects are obvious to see the differences 
  \item it is simple, the outlines do help to see depth
  \item Different colors make identifying objects' boundaries an easy task. 
  \item This visualization method has low complexity, easy to learn.
  \item The visulisation was easy to comprehend and I think I could easily figure out which blue orb was closer
  \item it's easy to see the areas and how close they are. 
  \item It was easy to understand straight away
  \item I have no opinion
  \item The shape of the object is very clear.
  \item Halo helps readability and distinguishing between "particles"
  \item the outline helped with the depth a bit
  \item I felt confident that I could guage depth using the fadedness of the lines.
  \item Simpe, easy to read, easy to compare colours
  \item The scene was very clear and comfortable to view. The outlines defined the objects even when small and deep within the green objects.
  \item the outline made it easier to see the different layers and to determine how close the blob was to me
  \item it gives me a shape information which is helpful for depth perception.
  \item It was good for counting
  \item This the most preferred visualization because its clear and easier to compare both objects to decide depth perception
  \item I think that the white lines are good to recognize.
  \item The curve of the edge can support visibility
  \item it was well prepared 
  \item there's no much visual distraction that could happen in this visualization
  \item it was easy to tell objects apart
  \item outlines made it easier to see the red and green objects but so much for the blue objects.
  \item Not too complicated
  \item contouring helps with colorblindness
\end{itemize}
\subsubsection{I disliked this visualization because.}
\begin{itemize}
  \item it is really hard to tell which one is closer in ar, especially it's limit ur movement  and graphics were rendered in x ray 
  \item Some object are not obvius ti see the distane
  \item Some of the pairs felt the same
  \item Hard to distinguish depth level.
  \item sometimes, when the two objects are in the similar distance to me, then i need to move around to be able to see it. 
  \item I found it difficult to interpret depth at times.
  \item It is hard to tell the depth of the objects and also hard to compare the depth of them. 
  \item I am not convinced the halo helps distinguishing depth levels but I guess my results will speak for themselves...
  \item Sometimes the headsets resolution made the lines look to pixel-y.
  \item Difficult to get a lot of contrast between shapes
  \item I don't think the outlines helped with the depth aspect very much.
  \item some of the outlines looked identical that it made it hard to determine which one was closer so i could only guess
  \item sometimes the outline reminds me a face, so it seems distracting.
  \item It was a lot harder in this experiment to determine which was closer. It made the usual method of how much green  is in front of the blue more difficult to determine I felt
  \item Probably less compatability when using Holo2 Lens with another glassess
  \item some of them were hard to distinguish whether right one is closer or the left one.
  \item it's similar to the basic "No Effect" visualization therefore had to rely on the color difference and other hint to tell the difference
  \item it was hard to gauge the distance of the objects
  \item Blue objects sometimes was hard to see it as 3D object
  \item Hard to tell the distance
  \item the depth perception was not enhanced as much than with the dots or wireframe visu
\end{itemize}
\subsection{Hatching}
\subsubsection{I liked this visualization because.}
\begin{itemize}
  \item That easy to see the objects' different
  \item Its the best one for describing shapes and it does help for depth perception a lot
  \item Defining the blue object made it easier to localize it
  \item This method is suitable to recognize 3D shape of tissue.
  \item It was very easy to figure out which blue object was closest because of the grid.
  \item it has the linein the objects which makes little bit easier to see the distance of the blue object 
  \item it's more clear than outline one
  \item Having the grid patterns made it easier to interpret depth than the 4th condition (standard visualisation).
  \item The cross hatching gives a good reference of comparing depth
  \item I think the parallel lines helped me "read" the depth
  \item Its easy to get an idea of how high  an object is.
  \item I could tell the shape of each blue thing really quickly and it made me much more confident in my answer.
  \item High contrast with crosshatching, nade discerning shapes easy
  \item The cross hatching gave a sense of the topology of the object which I think helped when comparing the blue objects against each other. 
  \item at first i thought it was easy to use as i would choose the grids that "pop out"more as the ones that are closer 
  \item it gives me some texture information of the object.
  \item The hatching effect allows me to differentiate depth perception of the objects clearer
  \item It looks intutive and quick to use
  \item The hashing made it easier to tell which was closer as it provided more 3d space definition. 
  \item this was a bit easier to distinguese between compare to past 2 tests.
  \item I could predict and estimate the depth based on the curve "angle" of the wireframe
  \item it was easy to tell objects apart
  \item Fun and vivid.
  \item I could aprehend the shapes better
\end{itemize}
\subsubsection{I disliked this visualization because.}
\begin{itemize}
  \item i don't like the x ray render texture on it 
  \item It can be a bit confusing. It would be awesome to be able to turn it on and off.
  \item Sometimes difficult to see inside tissue(blue) because of occlusion of visualization effects.
  \item no. 
  \item Due the the hatching on the outer objects, it added additional challenge to identifying the inner objects.
  \item It's slightly uncomfortable on the eyes, but didn't have any impact on my performance. However, I think if I had to use it for a longer duration (i.e. an hour) then it would start to probably impact my performance.
  \item It is lagging sometimes. 
  \item The angles of the lines move with the viewer's position, which is distracting
  \item It is visually overloading. Just a lot going on.
  \item The crosshatching made comparing depth more difficult. I ended up relying on comparing colour
  \item While I got used to the view-dependent effect, I think I would like it more if the grid effect stayed static on the object like a wireframe. There was also some artefecting when looking at the objects too close. 
  \item afterwards as i progressed through this visualisation, the grids that used to assist me made it more difficult to determine how close it is to me, i had to focus really hard and stare at it a lot to determine the smallest details to guess which one is closer.
  \item I felt like it oclude my eyesight sometimes.
  \item It is somehow frozen automatically
  \item The red blob would sometimes have hashing in front of the green/blue and cluttered it a bit. Not often though.
  \item visualizations were difficult to see or observe when there's too many wires on the wireframe
  \item didnt have any dislikes
  \item Compared to stippling the lines more heavier which made it bit more difficult to see,
  \item Nothing that I can think of. My favorite so far.
  \item it was a bit too much
\end{itemize}
\subsection{No VIRT}
\subsubsection{I liked this visualization because.}
\begin{itemize}
  \item n/a nothing good, just easy to use , nothing else then that
  \item The colour is obvius
  \item it is the cleanest but hard to tell depth
  \item This method does not cause mental effort and eye fatigue.
  \item simple and clean 
  \item The shape of the objects is very clear. 
  \item Simple, no distracting elements
  \item It was easier to tell the difference when the difference was smaller
  \item I could usually tell straight away which one was closer.
  \item Clean visuals made colour comparisons easy
  \item This was the clearest way to view the scene, and I found it emphasized the colour and shading of each object making it easier to use that as a cue for depth. I am unsure if that was more accurate or not however. 
  \item it was quite easy to determine which bob was closer to me, at first I just kept staring at both of them to see which one was closer, but afterwards i noticed that it would be easier if i just stared between the two blobs, it makes it way obvious
  \item it is intuitive and quite easier to understand.
  \item the vagueness of the edge can be easily observed
  \item Less clutter so I could focus on the green blob and how much I thought was closer to be compared to the blue blob
  \item it was nice to look for the objects in 3rd dimentions. 
  \item its simple
  \item didnt have any likes
  \item compared to the other ones, this one didnt have any lines or dots which made it a bit easier to look at.
  \item It's simpler
  \item it was simple
\end{itemize}
\subsubsection{I disliked this visualization because.}
\begin{itemize}
  \item i feel the xray texture is very disrupting 
  \item Some of the objests are really difficult to see the distance
  \item It is very difficult to asses depth withoute any guides
  \item I found it harder to perceive the blue object. I think the scene looked more planar (like clouds, there is depth that is hard to see) and my brain could not differentiate between the objects.
  \item Some of images impossible to percept depth.
  \item In most cases both the visualisations loked the same thus, it was difficult to figure out which one was the closest.
  \item somethings the transparent of the color confuses me, it make me feels like with high opacity is closer but sometimes it is not. 
  \item It was harder to visualise depth when  compared to other visualisations
  \item it's not clear
  \item I think that having the stippling/outline effects (cond 1/2) helped more with understanding depth.
  \item It hard to compare the depth of two objects. 
  \item Harder to guess depth without depth cues
  \item When I couldn't tell straight away, no ammount of moving my head made me feel any more confident in my guess.
  \item Difficult to discern relative depths between the two fields
  \item Without additional effects, I sometimes found it hard to determine the shape of objects when further behind the green containers as they were very faint. 
  \item at first when i kept staring atthe blobs it was hard to differenciate and the quality i was seeing was a bit low but afterwards once i got used to it, it was fine.
  \item I wasn't sure to find correct answer for some difficult questions. Not much visual information to decide the answer.
  \item This visualization was hard to use because the depth could not be differentiated. 
  \item It is difficult to judge the distance when the volume of both blue object is almost same with similar shade.
  \item When the blue blobs were close together is was almost impossible to tell without having some form of hashing or outline etc
  \item there isn't much clue or information that I could use to help with the depth perception
  \item found it hard to see object in other objets and to tell distances of objects
  \item At the same time, not having any white indicators made it hard to recongnize the object as 3D object. Sometimes it would be seen as a hole in the green object.
  \item Hard to tell the difference
  \item it was not obvious to determine which was closer
\end{itemize}
\subsection{Stippling}
\subsubsection{I liked this visualization because.}
\begin{itemize}
  \item the dot did helped me to didentiflied whch one is closer to me, and the system is easy to use 
  \item The colour is obvius to see
  \item It felt easier than the outline/halo condition. The dots help see parallax when moving to the sides and this makes easier to determine depth.
  \item It is much easier to determine how far is the blue object inside the green one.
  \item This method seems easy to learn and not too much affect to original image.
  \item The dots made it easy to understand the depth of the blue orbs
  \item it gives me more feeling of how blue object close. 
  \item Changes in the pattern could be discerned by distance
  \item the white dot is very helpful
  \item No idea why but I felt that depth was easier to interpret than the first condition (halo/outline).
  \item The dots on the surface give me a good reference to compare the depth
  \item Simple to understand and I feel like it gave me a good idea of the depth of the objects
  \item It was easy to visualize vertex points in 3d space
  \item I could almost always tell straight away which was the right one. I had a good idea of the shape of each blue thing.
  \item The white dots offered a point of contrast for comparison
  \item The "static" nature of the dots added extra detail that I felt made it easier to judge depth and shape. 
  \item the dots made it a bit easier to determine how close the blobs were to me
  \item it gives me clearer depth information.
  \item sometimes the bubbles helped determine one was closer
  \item Similar to hatching, the Stippling was easy to visualize depth of the objects
  \item The white dots help to understand the depth.
  \item The stips may be visually easy to count
  \item in this test it was way easier to identify the distance between both the blue objects. I was able to identify by the color density of the white dots and size of the white dots.
  \item the dots were useful for telling the depth perception
  \item didnt have any particular likes for the visualization 
  \item It looked very cool
  \item I think visualising things helps me to better understand things
  \item the paralax was easier to see
\end{itemize}
\subsubsection{I disliked this visualization because.}
\begin{itemize}
  \item the x ray texture is really disrupting 
  \item Some of the objects are difficult to see the differents
  \item The dots could get in the way of other info., but very effective.
  \item Little bit difficult to percept depth.
  \item sometimes, when the distances between objects and my eyes are similar, then it will be little bit diffcult to make decision 
  \item Depth is still hard for me to interpret.
  \item I like it
  \item The dots were a bit inconsistent
  \item Sometimes the dots looked a little cut off.
  \item The varying appearance of the dots (as in where they appeared on the green) obfuscated comparison to an extent
  \item While the dots were static, there was some artefacting during head movement that was slightly confusing. 
  \item i had to keep moving my head sideways to determine which one was closer and due to that it made me a bit dizzy in the process
  \item it was sometimes hard to discriminate shapes.
  \item But at the same time, sometimes they didn't feel as impactful. I mainly relied on how much green was layered over the blue
  \item It is less supportive than the glid or line.
  \item sometimes a cluster of dots could bring negative effect than helping
  \item it was sometimes hard to view overlapping objects 
  \item It was a bit hard to see things because it was bit blurry sometimes. It took a while for my eyes to adjust but it was fine afterwards. Looking at it for a long time might hurt my eyes a bit.
  \item Eyes gets exhausted in a while.
  \item the dots density didn't seem consistent
\end{itemize}

\section{Validation for Answers Given in the Post-Study Questionaire}
\subsection{What Made The Visualization Easy to use?}
\begin{itemize}
  \item The clarity of the nodes and visual was clear
  \item I can see all the objrcts obviusly
  \item it shows me the border/size i can tell where is it located at the volumn without interupt by colour
  \item The outline helped count green shapes and wasnt too confusing
  \item The white outline in the Halo condition made it really easy to spot the green objects
  \item Clear outlines, seperating the shapes from the volumes
  \item the outlines of the objects helped me to differenciate different objects even they are far away from me, i still can use the outline to see it. 
  \item In terms of use, the halo effect was the most logical and quick to understand. It did not make the scene more complicated and helped add definition. I would put no effect ahead of stippling as it is also simple and not overwhelming, even though the stippling was advantagous (see below). 
  \item the outline is clear
  \item I could pick out shapes easier with the outline and stippling. I could easily see shapes at the back aswell.
  \item The readability of the visualisations.
  \item No technical knowledge required. Task can be accomplished without learning any skill, it's just a matter of persistence really.

No effect and halo are the easiest, they feel comfortable and are aligned with our vision of the world every day. Stippling is not what we are used to be is still quite easy to use after using it for a while.
  \item The halo and no effect were easier as they did not hinder the visual differences between the object, hence improving clarity and allowing me to identify each individual.
  \item The clear boundary of the objects makes me easy to recognize the shapes of them
  \item When visualization method can assist to ditinguish which one is closer.
  \item The outlines allowed me to easily tell overlapping objects apart from each other
  \item Outline made it obvious how many green blobs there were from various angles 
  \item simplicity, lightlights blobs behind others (halo)
  \item outline/halo easily slows me which are the blobs and all i had to do was count
  \item The outlines were helpful to discriminate the object so i could easity count the number.
  \item It's easy to figure out the shape of green object.
  \item It was easier to count green objects on the no effect because there was no noice at all and I was able to identify objcts based on color density.
  \item the material of the objects and the ability to deduce the 3D shape.
  \item Reduced the amount of head movement to attain the same level of certainty before vlidating the result
\end{itemize}
\subsection{What Made The Visualization Difficult to use?}
\begin{itemize}
  \item The complexity of the image displayed did not make counting easy
  \item Almost all the objects are difficult to see and count
  \item the cros hatch become an interuption while i trying to determine the postion of that object, sometimes i try to find out is there any object behind other object, the cross hatch block my sign and cause confusion
  \item The cross Hatching was extremely confusing with a lot of shapes
  \item The grid in the cross hatching condition were covering the green objects making them dificult to spot
  \item No outlines, making the shapes difficult to discerne against long views of the volume
  \item too much overlapping with each other, and the far, the color is too transparent, which makes the objects are even harder to see it. 
  \item The cross-hatching visualisation made the task much harder as the effect was very dense and moved with my view, which wasn't "realistic" and made understanding the visualisation much harder. I would say any of the dense effects added "more" to parse for the user. 
  \item too much elements and hard to count
  \item Cross hatching made it hard to see shapes.
  \item Stippling \& Cross Hatching made it more difficult for me to identify whether close objects together was a single object or actually separate multiple objects. Cross-hatching was by far the worst because it was causing me physical discomfort when I moved my head too much.
  \item Mainly occlusion. When the green thingies are deep inside the volume, or when they are behind each others, they can be very hard to see.
The stippling helps a lot with that but caused (for me) a slight discomfort at the beginning. It goes away after a while.
Cross-hatching feels very weird and uncomfortable. I could not get used to it. On top of that, it was hard to see the added value. Counting was the hardest with it, because of what I mentioned before but also because sometimes it would "merge" green thigies depending on the angle of view.
  \item The stippling and cross hatching were present on all objects and made it harder to identify between each object, specifically with the cross hatching blocking the identification of objects beind it.
  \item The objects in the depth will faded in the backgruond color. 
  \item Some visualization method have too much visual effect.
  \item Lack of guidance and very clustered
  \item Crosshatching helped with overlapping but it was really noisy. I think if the red blob itself wasn't crosshatched it would be a better. It did help with overlapping blobs sometimes though
  \item Clutter, some vis techniques just added too much clutter
  \item there were many distractions in cross hatching, there were so many lines that it made it hard to see if there was a lightly coloured blob or even any blobs at all, so i had to count, and keep asking myself, "does that count as a blob?" and just guess
  \item The visualiztion seems to much so it made harder to perceive depth information.
  \item The boundary is blurred.
  \item cross hatching and stippling were a bit hard to work with because there too many lines or dots on the ojects which were creating confusions sometimes.
  \item I think Cross Hatching made the scene complicated so it was hard to count because there was alot of things (lines, specularity). The No effects lacks depth information so it is harder to locate the realtive depth of objects and also it is hard to seperate nearby objets.
  \item The depth perception needed "parala' help with me moving my head
\end{itemize}
\subsection{Why Do You Think You Performed Better With These Visualization?}
\begin{itemize}
  \item The outline effect was best because it was clear to count the nodes in both cases
  \item Because I can count the objects
  \item becuse i can see the border of the object clerly so i can dtermine it's postion
  \item I felt the most confident with Outlines and the No effect.
  \item In the Halo condition it was really easy to locate the green objects that's I think I could accurately count the number of the green objects
  \item Easy identification of small shapes. Clear identification of shapes within the volume
  \item i can clearly see and differenciating different objects. 
  \item Both the halo and stippling effects helped me perform better becuase they added additional information to the objects which assisted where objects where occuluded or deep. The halo effect felt better to use as the white colouring seemed stronger with depth, whereas the stippling still required me to move around a lot more to detirmine overlapping objects apart.
  \item the visulization is clear
  \item I could keep keep track of shapes with the outline and stippling.
  \item I think no effect was best because it was overall easier on me to mentally keep track of the green objects when counting. I felt like I made alot more mistakes (For example - forgetting whether I had already counted an object and having to start over) in comparison to the other visualisations. I believe Outline was the best for depth-perception.
  \item I think halo was good but not as effective in terms of helping counting the green thingies than the stippling one. Halo just provides a contour, so you it highlights the separate green thingies really well, but stippling's random dots sometime reveals occluded green thigies I might not have seen otherwise.
As much as I hated the cross hatching, I still think it helped me count compared to the no effect condition.
  \item Halo and no effect as I believe it was easier to identify and hence count each object
  \item I can finish the counting faster. It is easier for me recognize the shapes and locations of the objects. 
  \item The Outlines \textbackslash Halo visualization is most comfort interms of mental effort and esay to count total number of green objects.
  \item Felt it was easier to count
  \item Outline made me more confident. No effect there was less noise so I could focus.
  \item Same as above
  \item I was more confident when the blobs were specifically outlined, it made counting much easier
  \item The outlines highlights the object's surface so I could answer faster with confidence.
  \item The boundaries were clear and easy to count
  \item with outlines I was easily able to count the number of green objects there was not much going on I was able to recognise if the object is inside the blue object or outside.
  \item I think I had no difficulty with the two questions under the stiplling.
  \item The easier the task seems to me, the better is my percieved performance
\end{itemize}
\subsection{Why Do You Think You Performed Worse With These visualization?}
\begin{itemize}
  \item The clarity was poor
  \item I cannot see the objects
  \item becasue the cross hatch cause interuption hwile i trying to locate the object , especially hwne the object is behind other object 
  \item I got mostly confused with these stippling and cross hatching
  \item The grid in the cross hatching condition made it really difficult to locate the green objects thus, I feel i couldn't count accurately
  \item Difficult to discerne shapes, worse performance made it harder to pay attention too
  \item cannot easily differenciating the objects which might cause double counting or miss counting 
  \item The cross hatching effect was view dependent which did not feel natural and I don't think helped determine object depth or positions. This caused me to want to look closer, but this would cause the effect to become heavier in my mind. 
  \item it's hard to count 
  \item I had more trouble remembering which shapes I had already counted with "no effect" and cross hatching.
  \item I think I performed reasonably poor with stippling/cross-hatching due to readability and mental discomfort.
  \item I believe that no effect makes it really difficult distinguish two green thingies that are close together, but also two green thingies occluding each others. Cross hatching causes lots of headache; feels like there was too much information that was barely useful in the end. The "merging" effect I mention in the previous question also makes it difficult to distinguish between two green thigies that are too close together.
  \item Stippling and cross hatching as I found it harder to identify these and sometimes felt like I was guessing 
  \item It is hard to tell the location and the shapes of the objects, I need more time to double confirm. 
  \item Too difficult to count grouped objects and hard to percept depth.
  \item Felt like i struggled with them due to clustered infromation
  \item Crosshatch was too noisy and made me less focused. Stippling to a lesser extent but it wasn't that bad. I just feel I could focus more on the no effect and outline.
  \item Same as above
  \item i couldn't determine if what i saw was classified as a green blob making it very hard to count
  \item It was quite demanding for me to count the number in the cross hatching condition.
  \item It was hard tounderstand if the green object was one or several attached to it.
  \item cross hatching has too many lines which was confusing me on green objects whihc are at the back of the whole volume. 
  \item I needed more cues to deduce the depth. Parallax was not always enough.
  \item The easier the task seems to me, the better is my percieved performance
\end{itemize}

\section{General Comments}
\begin{itemize}
  \item The display kept dropping in and out hence it made counting harder as I would have to recount again. 
  \item Some of the effects are easy to count
  \item Thank you
  \item Very cool tech, good work
  \item the outline one is the best makes everyting easier. Would prefer to increase the transparency of the cross hatching line for the red objects and reduce the transparency of the green or blue cross hatching, would be better. Also would be better to remove the billboard functions for the cross hatching, I would prefer the objects are staying not follow my eyes to rotate, so i can see the objects from different views. 
  \item I think FPS performance is a factor as with the heavier effects (i.e. crosshatching) the object would jump and jitter a little more, which when combined with the view dependent effect was a bit confusing. 
  \item I never really had a good strategy to do this task, for most of  the task I was just trying to mentally keep track of what objects I had already counted. I would start from a random object and then go to the next closest object etc and keep track that way. Early on in the study I tried to go in order from top to bottom or furthest to closest but this strategy wasn't working working well for me
  \item Not that I can think of right now.
  \item The cross harching one is really lagging and make me feel sick. 
  \item Mainly just crosshatching looks like it could be great as it helps detect overlaps. But it's very noisy with crosshatching the red blob (I assume it is anyway). That or there is a lot of other blobs inside the red blob and it just gets too cluttered with crosshatching for me.
  \item counting outloud and pointing made it easier for me to count, cause when i move my head, the headset moves along which made me dizzy, pointing keeps me focused and eases the dizziness
  \item Other than "No effect", the visulization was not stable when I move my head quickly. It may affect the ease of use (it may also get people dizzy) but I personally tried to ignore it.
  \item I was able to se clearly thgough the green volumes with the No effect condition, which helped me spotting the background objects
\end{itemize}

\section{Validation for Answers Given in the Post-Study Questionaire}
During the post hoc comparison, participants were asked a series of questions stating why they selected some answers over others. In terms of what was easy vs what what they believed they did better with.

\subsection{What Made The Visualization Easy to use?}
\begin{itemize}
  \item Because the distance and colour are obvius
  \item For some reason the dots immediately made sense and helped a lot with depth perception.
  \item Stippling method does not much affect to original image and easy to recognize which one is closer.
  \item the distances between the lines are easier for me to judge, i used that as the refences to make the decision. 
  \item the white dot
  \item It has clear references to compare the depth of them. 
  \item Stippling is quite discreet but yet does it a good job (IMO) at indicating the depth. Halo also seemed to help for some reason (might be placebo) but the reason why I preffered it over the others is because it's just simple to read and appealing.
  \item Had points where its easy to tell the depth
  \item Stippling and cross hatching let me tell the shape of the blue thing very quickly, but stippling was more consistent when I moved my head.
  \item colour contrast as a depth indicator
  \item I think the stippling and outline effects were the easiest to use as they were quick to understand and didn't experience artefacting when moving my head. 
  \item Stippling and CH allowed for a better understand of front and back
  \item the visualizations were consistent which made it easier for me to use
  \item The glittering surface made me easier to perceive depth information.
  \item The outline/halo effect was clearer to differentiate the depth of different objects
  \item The dots and lines are helpful.
  \item The Halo/Outline can quickly help detect the distance by its quantities
  \item Hatching allowed for an indication when the blue blobs were close together so I found it easier
  \item with stippling it was very easy to identify the closest blue object due to color density of white dots and size of the dots in both the objects. I was also able to identify the shape of both of the blue objects due to different orientation of white dots. 
  \item The dots in stippling helped me to better understand the closest object easier than the other three visualizations
  \item the objects visualization made it more understanding when judging distances
  \item No effect was easy to see but hard to understand the 3D object.
  \item Seems more vivid and 3D
  \item I was primarily focusing on the blurryness of the blue object, the stipling and cros hatching helped to emphasize on the object paralax
\end{itemize}
\subsection{What Made These visualization Difficult to use?}
\begin{itemize}
  \item The distance is difficult to identify
  \item Because there are no reference points. trying to figure out depth of semitransparent objects is always hard.
  \item If visual effect is too much used, it might cause mental load and difficult to percept depth since occlusion of outer tissue's visualization effect.
  \item In many cases, i have to move around to be able to see which object is closer. 
  \item there is no outline and hard to check the depth of each object
  \item It does not have clear references to compare the depth 
  \item Cross Hatching, if it's supposed to work like I think it's supposed to, is in theory a great idea. However it is very hard to distinguish between small differences in depth, on top of being very distracting because it changes based on the viewer's location.
No effect was definitely the hardest because appart from the decrease in saturation depending on the depth and the perspective, there is nothing to indicate depth.
  \item Covering the objects made it difficult to compare depth. I think if there is too much going on then it makes it harder not easier.
  \item If I couldn't tell early on which shape was closer, no effect and halo (in part) did not help much when I moved my head.
  \item obfuscation of depth from additional components, ie, crosshatching
  \item The cross hatching was the most confusing simply because of the way it changes based on head position, wheras a static wireframe would have been easier.
  \item None cause an optical allusion that made the red and green parts of the vis move, Halo provided not depth understanding
  \item because i have no understanding as to how to differenciate the closeness of blob with these visualizations, so i had to get used to it in the beginning 
  \item I don't have much visual information so I had to back and forth to decide.
  \item No effect was the only least preferred visualization as it was harder to identify the object's distances
  \item It's difficult to understand the depth because the boundaries are blurred.
  \item There is little highlighted marks so that it might require users to come up with other strategy to identify the distance like the volume/brightness/blurriness etc.
  \item Outline did not improve how close they were and it felt like the green blob was harder to determine how close it was to me. No effect itself I could focus a lot on it though but it was difficult to determine which was closest when they were closer together.
  \item No Effect was a bit difficult to use because I felt like both the images are dull and blur. so with that it was a bit hard to identify which one is close even after head movement.
  \item there isn't too much visualization or effect to support my task
  \item was hard to tell distances and see overlapping objects
  \item For Cross Hatching and Stippling, it was a bit difficult to see but made it easier to recognize the bkue objects as 3D objects.
  \item Too flat
  \item I had little to no depth information
\end{itemize}
\subsection{Why Do You Think You Performed Better With These visualization?}
\begin{itemize}
  \item Becsue I can see the distance and all the colour of the object
  \item Cross hatching does help a lot when you take a bit of time to move your head to the sides. i guess this was my best score.
  \item Easy to understand purpose of visualization.
  \item not sure, based on my guess
  \item the white dot can help to check the object's depth
  \item It has clear references, the shape of the objects are clear. 
  \item Same reason as before
  \item The halo did help a bit but it was still a bit harder than no effect.
  \item Stippling was less annoying than cross hatching so I felt more confident with my answer quicker.
  \item practice, high colour contrast for comparison
  \item The stippling effect felt like it enhanced my ability to understand the 3D depth of the obejcts and wasn't too noisy. I'm not sure if I actually did well with no effect, but it was easier to focus on just the colour of the object with no other white visualisation effect. 
  \item See above
  \item No effect had the least "distractions" although the other visualizations were to assist me in making a decision, i also didnt have to move much for no effect while i had to keep moving my head position for the others which made it dizzy for me
  \item I think I relatively easily decide the answer in the stippling condition.
  \item Personally, I felt more confident in judging the depth of the objects
  \item I think that it's easy to understand.
  \item The Outlines/Halo condition ease the most consideration workload when making selections
  \item Hatching I could tell the close together ones easier. No effect I could focus on the green blob size so I think I got a majority of the easier ones correct without second guessing. But was hard to tell the close together ones
  \item It was much eaier to identify the shape and size of the blue objects with stippling effect. I was also able to identify the distance from me with the help of color density. I did not have to move much I was able to identify by sitting still. some of them required me movement because some of them looked kinda similar (90% similar).
  \item There are many attributes from this visualization that helped me with the prediction "color, size, the dots"
  \item was easier to tell objects apart
  \item  I used light reflections to recognize the objects but even then a bit of white visualization was necessary to see it as 3D object. But too much visualization made it hard to see.
  \item I am more confident because it is easier to tell the distance.
  \item same as previous question
\end{itemize}
\subsection{Why Do You Think You Performed Worse With These visualization?}
\begin{itemize}
  \item I coudnt see the distance of the object
  \item For the 'no effect' condition I was mostly guessing. So I imagine this was by worst score.
  \item I just had to rely on brightness and contrast of tissue, it might not indicate its depth.
  \item the objects are not easy to see it clearly for the distance
  \item it's not clear
  \item Very hard to compare the depth of the objects. Doesn't have reference in the environment
  \item Same reason as before
  \item Too much going on (as well as covering the objects)
  \item I think I felt like I was guessing too much with no effect.
  \item obfuscation of depth and difficulty comparing colour
  \item The cross hatching was a close second/third but I think due to the movement I never felt fullly confident with it. However it did help the most with understanding the shape of the objects. Finally I think the outline was the worst because even though it was very clear to understand, the outline made it harder to focus on the colour and felt like it exaggerated depth rather than make it feel clearer. 
  \item See above
  \item i found it rather distracting and i had to think more to determine, and staring at the strippling and crosshatching made me a bit dizzy and had to keep reminding myself "which is closer"
  \item It was most difficult when I had to choose the answer in the no effect condition.
  \item The no effect was essentially harder to use as there was no way to identify the depth of the objects
  \item There were times when I was a little unsure.
  \item the cross Hatching hightlighted the edge but hinder the judgement of distance
  \item Outline just felt difficult to determine in 3d space, made it more 2d style so I feel I got a lot incorrect.
  \item I found no effect the worst because I was not able to focus on the blue object. it was blurry most of the time. I had to move to much to identify the real closest blue objects but I may have failed on some of them.
  \item There are too much distractions in the cross hatching "or wireframe"
  \item was hard to tell ojects apart and gauge distances
  \item Having no virtualization made it harder to see.
  \item Had a few guess.
  \item same as previous question
\end{itemize}


\section{General Comments}
The following comments were gathered from users as a method for them to provide general feedback about the study or to say whatever they wanted to say. 
\begin{itemize}
  \item If I can face to a white wall of the background will be better
  \item Thank you!
  \item One idea is allocate artificial spot light and shadow. 
  \item I'm not sure if you added it or not, but i used the transparency of the color to help to judge the locations. 
  \item The cross hatching one is bit of lagging. 
  \item Interesting study!
  \item I think if cross hatch was more consistent when I moved my head, I might have liked it more.
  \item fun study
  \item I think my answers here are accurate for a sitting experience wher one cannot move around the visualisation to observe it. If I was able to use more signficant movement to help judge depth, then the outline effect would have been preferable I think. 
  \item The allusion caused by the none condition was interesting, it look although the green and red areas were slowly rotating.
  \item none
  \item What if the objects in each study were the same size? Would that change the participants choices in idenifying the depth of each object?
  \item Probably can add more guidance before the day so that paticipants may have more clear ideas on this research.
  \item the eye can get fatigue especially in the final condition due to the long usage "~1hour" of focus therefore I'm concerned if my performance could be negatively affected in the last condition. In addition, the reflection from the HoloLens 2 made me easy to see the researcher therefore causing distruptions at some point. 
  \item Overall, my eyes felt bit strained starting for a long time but otherwise, it was fine.
\end{itemize}


\chapter{Statements of Authorship}
This section contains the Statement of Authorship clarifying the contributions of the authors of the published works found in this thesis.

\includepdf[pages=-,pagecommand{},width=\textwidth]{Appendix/StatementsOfAuthorship/AdaptingVSTARX-RayVisionTechniquesToOSTAR.pdf}
\includepdf[pages=-,pagecommand{},width=\textwidth]{Appendix/StatementsOfAuthorship/AdaptingVSTARX-RayVisionTechniquesToOSTAR-1.pdf}
\includepdf[pages=-,pagecommand{},width=\textwidth]{Appendix/StatementsOfAuthorship/GeneratingPseudoRandomVolumesForVolumetricResearch.pdf}
\includepdf[pages=-,pagecommand{},width=\textwidth]{Appendix/StatementsOfAuthorship/GeneratingPseudoRandomVolumesForVolumetricResearch-1.pdf}
\includepdf[pages=-,pagecommand{},width=\textwidth]{Appendix/StatementsOfAuthorship/SuperpowersInTheMetaverseAugmentedRealityEnabledXRayVisionInImmersiveEnvironments[1] (2).pdf}
\includepdf[pages=-,pagecommand{},width=\textwidth]{Appendix/StatementsOfAuthorship/SuperpowersInTheMetaverseAugmentedRealityEnabledXRayVisionInImmersiveEnvironments[1] (2)[1].pdf}
\includepdf[pages=-,pagecommand{},width=\textwidth]{Appendix/StatementsOfAuthorship/VolumetricXrayVisionUsingIllustrativeVisualEffects.pdf}
\includepdf[pages=-,pagecommand{},width=\textwidth]{Appendix/StatementsOfAuthorship/VolumetricXrayVisionUsingIllustrativeVisualEffects-1.pdf}
%\includepdf[pages=-]{myfile.pdf}