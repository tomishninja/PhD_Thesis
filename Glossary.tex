\newglossaryentry{SpaitalAwareness}
{
    name=Spatial awareness,
    plural=Spatial awarenesses,
    description={The ability to perceive and understand the spatial relationships between objects in the environment, as well as the position and movement of one's own body in relation to those objects. It is a fundamental aspect of human perception and is essential for many activities, such as navigation, manipulation of objects, and interaction with others.}
}

\newglossaryentry{Depth Perception}
{
    name=Depth Perception,
    plural=Depth Perceptions,
    description={The ability to perceive and judge the spatial relationships between objects in three dimensions. An essential aspect of human vision, allowing us to navigate and interact with the physical environment.}
}

\newglossaryentry{X-ray Visualization}
{
    name=X-ray Visualization,
    plural=X-ray Visualizations,
    description={A rendering or graphical technique used in computer graphics, medical imaging, and augmented reality to simulate the effect of looking through solid objects. Instead of physically penetrating materials as in real X-rays, this method selectively hides, fades, or highlights parts of a model or environment to reveal obscured internal structures or layers for analysis, learning, or interaction.}
    %description={A graphical effect that gives the user the impression of seeing through real-world objects.}
}

\newglossaryentry{X-ray Vision}
{
    name=X-ray Vision,
    description={A perceptual or technological capability, either fictional or simulated, that allows an observer to see through opaque objects and reveal their hidden interior structures or the environment beyond them. In practice, X-ray vision is achieved through imaging technologies or visual effects that mimic this ability, providing insights not accessible through direct surface observation.}
    %description={The ability to see through solid objects and view the object’s internal structures or the environment beyond it.}
}

% \newglossaryentry{xrayvisualization}{
%     name={X-ray Visualization},
%     plural={X-ray Visualizations},
%     description={A rendering or graphical technique used in computer graphics, medical imaging, and augmented reality to simulate the effect of looking through solid objects. Instead of physically penetrating materials as in real X-rays, this method selectively hides, fades, or highlights parts of a model or environment to reveal obscured internal structures or layers for analysis, learning, or interaction.}
% }

% \newglossaryentry{xrayvision}{
%     name={X-ray Vision},
%     description={A perceptual or technological capability, either fictional or simulated, that allows an observer to see through opaque objects and reveal their hidden interior structures or the environment beyond them. In practice, X-ray vision is achieved through imaging technologies or visual effects that mimic this ability, providing insights not accessible through direct surface observation.}
% }

\newglossaryentry{AugmentedRealityX-rayVision}
{
    name=Augmented Reality enabled X-ray Vision,
    plural=Augmented Reality enabled X-ray Visions,
    description={The ability to overlay virtual content on top of the real-world environment, allowing users to see virtual objects positioned behind real-world obstacles.}
}

\newglossaryentry{microsaccades}
{
    name=Microsaccades,
    text=microsaccades,
    description={Microsaccades are tiny, involuntary eye movements that occur even when we try to maintain fixation on a point, serving important functions in visual perception and attention by preventing sensory adaptation and aiding in gathering additional visual information from the environment.}
}

\newglossaryentry{ar_g}{
    name= {Augmented Reality},
    plural= {Augmented Reality},
    description={A technology that overlays digital information and virtual objects onto the real world, enhancing the user's perception and interaction with their environment through a smartphone or smart glasses device. By blending the physical and digital realms, AR provides immersive experiences and practical applications across various industries, ranging from entertainment and gaming to education and industrial design.}
}

\newglossaryentry{ar}{
    type=\acronymtype, 
    name = {AR\glsadd{ar_g}},
    description={Augmented Reality\glsadd{ar_g}}, 
    plural={AR},
    first={Augmented Reality (AR)\glsadd{ar_g}}, 
    see=[Glossary:]{ar_g}
}

\newglossaryentry{vr_g}{
name= {Virtual Reality},
plural={Virtual Reality},
description={A technology that creates a simulated and immersive digital environment, typically experienced through a head-mounted display and motion-tracking devices. It transports users to a computer-generated world, allowing them to interact with realistic or fantastical scenarios, offering a range of applications from entertainment and gaming to training and therapy.}
}

\newglossaryentry{vr}{
type=\acronymtype,
name = {VR},
plural = {VR},
description = {Virtual Reality},
first={Virtual Reality (VR)\glsadd{vr_g}},
see=[Glossary:]{vr_g}
}

\newglossaryentry{ost_g}{
name= {Ocular See Through},
plural = {Ocular See Through},
description={A technology that provides a transparent display, allowing users to view the real world while overlaying digital information or virtual objects onto their field of vision. By enabling the simultaneous perception of the physical and digital environments, OST enhances user experiences and finds applications in various fields, including gaming, navigation, and industrial training.}
}

\newglossaryentry{ost}{
type=\acronymtype,
name = {OST},
plural = {OST},
description={Ocular See Through},
first={Ocular See Through (OST)\glsadd{ost_g}},
see=[Glossary:]{ost_g}
}

\newglossaryentry{vst_g}{
name= {Video See Through},
plural= {Video See Through},
description={A technology that utilizes a video feed from cameras to overlay digital information or virtual objects onto the real world, allowing users to see their surroundings through a display device. By merging live video footage with computer-generated graphics, VST enhances user perception and enables interactive experiences in fields such as gaming, medical imaging, and remote assistance.}
}

\newglossaryentry{vst}{
type=\acronymtype,
name = {VST},
plural = {VST},
description={Video See Through},
first={Video See Through (VST)\glsadd{vst_g}},
see=[Glossary:]{vst_g}
}

\newglossaryentry{dvr_g}{
name= {Direct Volume Rendering},
plural={Direct Volume Rendering},
description={A technique in computer graphics and visualization that allows for the direct generation of images from volumetric data, such as medical scans or scientific simulations. DVR enables the exploration and visualization of complex three-dimensional structures and phenomena by applying various rendering algorithms and shading techniques.}
}

\newglossaryentry{dvr}{
type=\acronymtype,
name = {DVR},
plural = {DVR},
description={Direct Volume Rendering},
first={Direct Volume Rendering (DVR)\glsadd{dvr_g}},
see=[Glossary:]{dvr_g}
}

\newglossaryentry{lmm_g}{
name= {Linear Mixed Effects Models},
description={A statistical modeling approach that incorporates both fixed effects and random effects to analyze data with hierarchical or clustered structures. LMMs are used to account for within-group dependencies and capture individual variations, making them suitable for studying longitudinal or repeated measures data in various fields, including social sciences, biology, and psychology.}
}

\newglossaryentry{lmm}{
type=\acronymtype,
name = {LMM},
plural = {LMMs},
description={Linear Mixed Effects Models},
first={Linear Mixed Effects Models (LMM)\glsadd{lmm_g}},
see=[Glossary:]{lmm_g}
}

\newglossaryentry{hsd}{
    type=\acronymtype,
    name = {HSD},
    plural = {HSD},
    description={Honestly Significant Difference},
    first={Honestly Significant Difference (HSD)}
}

\newglossaryentry{hmd_g}{
name= {Head Mounted Device},
description={Technological devices that are worn on the head and typically feature a display screen, enabling users to experience immersive visual and auditory content. Head Mounted Devices, such as virtual reality headsets or augmented reality glasses, provide a user-centric interface and are widely used for gaming, virtual simulations, training, and entertainment purposes.}
}

\newglossaryentry{hmd}{
type=\acronymtype,
name = {HMD},
plural = {HMDs},
description={Head Mounted Device},
first={Head Mounted Device (HMD)\glsadd{hmd_g}},
firstplural={Head-Mounted Displays (HMDs)\glsadd{hmd_g}}
see=[Glossary:]{hmd_g}
}

\newglossaryentry{ct_g}{
    name= {Computed Tomography},
    plural= {Computed Tomographies},
    description={A medical imaging technique that uses X-ray technology to create detailed cross-sectional images of the body. Computed Tomography (CT) scans provide valuable diagnostic information by capturing multiple X-ray images from different angles and combining them to generate a three-dimensional representation. CT scans are commonly used to diagnose various conditions and guide medical procedures.}
}

\newglossaryentry{ct}{
    type=\acronymtype,
    name = {CT},
    plural = {CTs},
    description={Computed Tomography},
    first={Computed Tomography (CT)\glsadd{ct_g}},
    firstplural={Computed Tomographies (CTs)\glsadd{ct_g}},
    see=[Glossary:]{ct_g}
}

\newglossaryentry{mri_g}{
    name= {Magnetic Resonance Imaging},
    plural={Magnetic Resonance Imagings},
    description={A medical imaging technique that uses strong magnetic fields and radio waves to create detailed images of the internal structures of the body. Magnetic Resonance Imaging (MRI) provides high-resolution images that help diagnose various conditions and assess the health of organs and tissues. MRI scans are widely used in medicine for their non-invasive nature and ability to visualize soft tissues with excellent contrast.}
}

\newglossaryentry{mri}{
    type=\acronymtype,
    name = {MRI},
    plural = {MRIs},
    description={Magnetic Resonance Imaging},
    first={Magnetic Resonance Imaging (MRI)\glsadd{mri_g}},
    firstplural={Magnetic Resonance Imagings (MRIs)\glsadd{mri_g}},
    see=[Glossary:]{mri_g}
}

\newglossaryentry{sus_g}{
    name= {System Usability Scale},
    plural={System Usability Scales},
    description={A widely used questionnaire-based method for assessing the usability of a system or product. The System Usability Scale (SUS) consists of a set of standardized statements and Likert scale responses that provide quantitative measures of perceived usability, effectiveness, and user satisfaction. SUS scores are commonly used in user experience research to evaluate and compare the usability of different systems or product designs.}
}

\newglossaryentry{sus}{
    type=\acronymtype,
    name = {SUS},
    plural = {SUS},
    description={System Usability Scale},
    first={System Usability Scale (SUS)\glsadd{sus_g}},
    firstplural={System Usability Scales (SUS)\glsadd{sus_g}},
    see=[Glossary:]{sus_g}
}

\newglossaryentry{sar_g}{
    name= {Spatial Augmented Reality},
    plural={Spatial Augmented Realities},
    description={A technology that enhances the physical environment by projecting digital content onto real-world surfaces, such as walls, floors, or objects. Spatial Augmented Reality (SAR) combines projection mapping techniques with computer vision and spatial tracking to create interactive and immersive experiences. SAR finds applications in areas such as art installations, architectural visualization, and interactive displays.}
}

\newglossaryentry{sar}{
    type=\acronymtype,
    name = {SAR},
    plural = {SAR},
    description={Spatial Augmented Reality},
    first={Spatial Augmented Reality (SAR)\glsadd{sar_g}},
    firstplural={Spatial Augmented Realities (SAR)\glsadd{sar_g}},
    see=[Glossary:]{sar_g}
}

\newglossaryentry{imu_g}{
    name= {Inertial Measurement Unit},
    plural={Inertial Measurement Units},
    description={A sensor module that combines multiple inertial sensors, such as accelerometers, gyroscopes, and magnetometers, to measure and track an object's motion and orientation in real time. Inertial Measurement Units (IMUs) are commonly used in applications that require motion sensing, such as robotics, virtual reality, and motion capture systems. IMUs provide essential data for estimating objects' position, velocity, and orientation changes.}
}

\newglossaryentry{imu}{
    type=\acronymtype,
    name = {IMU},
    plural = {IMUs},
    description={Inertial Measurement Unit},
    first={Inertial Measurement Unit (IMU)\glsadd{imu_g}},
    firstplural={Inertial Measurement Units (IMUs)\glsadd{imu_g}},
    see=[Glossary:]{imu_g}
}

\newglossaryentry{dicom_g}{
    name= {Digital Imaging and Communications in Medicine},
    plural={Digital Imaging and Communications in Medicines},
    description={A standard for managing, storing, and transmitting medical images and related data. DICOM (Digital Imaging and Communications in Medicine) enables interoperability between different medical imaging devices and systems, ensuring compatibility and consistent communication of medical information. It is widely used in healthcare settings to store and exchange medical images, such as X-rays, CT scans, and MRIs.}
}

\newglossaryentry{dicom}{
    type=\acronymtype,
    name = {DICOM},
    plural = {DICOM},
    description={Digital Imaging and Communications in Medicine},
    first={Digital Imaging and Communications in Medicine (DICOM)\glsadd{dicom_g}},
    firstplural={Digital Imaging and Communications in Medicines (DICOM)\glsadd{dicom_g}},
    see=[Glossary:]{dicom_g}
}

\newglossaryentry{pc_g}{
    name= {Personal Computer},
    plural={Personal Computers},
    description={A type of computer designed for individual use, typically consisting of a central processing unit (CPU), memory, storage, and input/output devices. Personal computers (PCs) are widely used for various purposes, including work, communication, entertainment, and personal productivity. They offer a flexible and customizable computing platform for users to perform tasks, run software applications, and access the internet.}
}

\newglossaryentry{pc}{
    type=\acronymtype,
    name = {PC},
    plural = {PCs},
    description={Personal Computer},
    first={Personal Computer (PC)\glsadd{pc_g}},
    firstplural={Personal Computers (PCs)\glsadd{pc_g}},
    see=[Glossary:]{pc_g}
}

\newglossaryentry{hu_g}{
    name= {Hounsfield Units},
    plural={Hounsfield Units},
    description={A unit of measurement used in computed tomography (CT) imaging to quantify the radiodensity of tissues. Hounsfield Units (HU) assign a numerical value to tissue attenuation, where higher values represent denser or more radio-opaque structures and lower values indicate less dense or more radio-lucent structures. HU values help in differentiating and characterizing various tissues and abnormalities in medical imaging.}
}

\newglossaryentry{hu}{
    type=\acronymtype,
    name = {HU},
    plural = {HU},
    description={Hounsfield Units},
    first={Hounsfield Units (HU)\glsadd{hu_g}},
    firstplural={Hounsfield Units (HU)\glsadd{hu_g}},
    see=[Glossary:]{hu_g}
}

\newglossaryentry{mr_g}{
    name= {Mixed Reality},
    plural={Mixed Realities},
    description={A technology that merges elements of both the physical and virtual worlds, creating a seamless and interactive environment. Mixed Reality combines aspects of both Augmented Reality (AR) and Virtual Reality (VR), allowing users to interact with digital objects while maintaining awareness of the real world. This technology finds applications in fields such as gaming, architecture, training, and collaborative workspaces.}
}

\newglossaryentry{mr}{
    type=\acronymtype,
    name = {MR},
    plural = {MR},
    description={Mixed Reality},
    first={Mixed Reality (MR)\glsadd{mr_g}},
    firstplural={Mixed Realities (MR)\glsadd{mr_g}},
    see=[Glossary:]{mr_g}
}

\newglossaryentry{caves_g}{
    name= {Cave Automatic Virtual Environment},
    plural={Cave Automatic Virtual Environments},
    description={Cave Automatic Virtual Environment, or CAVE, is an immersive virtual reality environment that utilizes multiple large displays and surround sound to create a fully immersive experience. Users typically stand within a room-sized cube, and the walls, floor, and ceiling are used as screens to display virtual content. A CAVE provides a highly immersive and interactive virtual reality experience, enabling users to navigate and interact with digital environments naturally and intuitively. It finds applications in fields such as scientific visualization, architectural design, and medical research.}
}



\newglossaryentry{cave}{
    type=\acronymtype, 
    name = {CAVE},
    plural = {CAVEs},
    description={Cave Automatic Virtual Environment}, 
    first={Cave Automatic Virtual Environment (CAVE)\glsadd{caves_g}}, 
    firstplural={Cave Automatic Virtual Environments (CAVEs)\glsadd{caves_g}},
    see=[Glossary:]{caves_g}
}

\newglossaryentry{ultrasound}{
    name= {Ultrasound},
    plural={Ultrasounds},
    text= {ultrasound},
    description={A medical imaging technique that uses high-frequency sound waves to visualize internal structures of the body. Ultrasound imaging, also known as sonography, produces real-time images by emitting sound waves and capturing their reflections off tissues and organs. It is commonly used for non-invasive imaging of various body parts, such as the abdomen, heart, and developing fetus, and plays a crucial role in diagnosing medical conditions.}
}

\newglossaryentry{xray}{
    name= {X-ray},
    plural={X-rays},
    description={A form of electromagnetic radiation commonly used in medical imaging to visualize the body's internal structures. X-ray imaging involves passing X-ray photons through the body, with denser tissues absorbing more photons, resulting in varying levels of brightness in the final image. X-rays are widely utilized for diagnosing bone fractures, detecting abnormalities, and examining structures like the chest, teeth, and bones.}
}

\newglossaryentry{gantry}{
    name= {Gantry},
    plural={Gantries},
    text= {gantry},
    description={The circular or cylindrical structure of a CT scanner that houses the X-ray tube and detector array. The gantry rotates around the patient during the scanning process, capturing X-ray images from various angles. It plays a crucial role in producing cross-sectional images of the body, which are then used to reconstruct detailed 3D images. The design of the gantry allows for precise positioning and accurate imaging during CT scans.}
}

\newglossaryentry{isosurface}{
    name= {Iso-surface},
    plural={Iso-surfaces},
    text= {iso-surface},
    description={A three-dimensional representation of a specific value or range within a volumetric dataset, often used in scientific visualization. An iso-surface is created by identifying voxels (3D pixels) within the dataset that match a certain threshold value and then rendering the surface that connects these voxels. Iso-surfaces are useful for visualizing structures and boundaries within complex datasets, such as medical imaging scans or simulations in various scientific fields.}
}

\newglossaryentry{pathology}{
    name= {Pathology},
    plural={Pathologies},
    text= {pathology},
    description={The medical specialty that involves the study and diagnosis of diseases by examining tissue samples, cells, and bodily fluids. Pathologists analyze the structural and functional changes in tissues and organs to understand the nature of diseases and their underlying causes. Pathology plays a crucial role in disease diagnosis, treatment planning, and research, contributing to advancements in medical knowledge and patient care.}
}

\newglossaryentry{pathological}{
    name= {Pathological},
    plural={Pathologicals},
    text= {pathological},
    description={Relating to the study of diseases or disorders in living organisms. Pathological conditions involve abnormal changes in structure or function that can lead to health issues. Pathological studies aim to understand diseases' causes, mechanisms, and effects, often involving laboratory analysis, imaging, and clinical observations.}
    see=[Glossary:]{pathology}
}

\newglossaryentry{ui_g}{
    name= {User Interface},
    plural={User Interfaces},
    description={The point of interaction between a user and a digital or mechanical device, system, or software application. User interfaces encompass visual, auditory, and tactile elements that allow users to interact with and control devices or software, making it easier for humans to operate and communicate with machines. Effective user interfaces are crucial for ensuring a positive user experience and optimizing usability.}
}

\newglossaryentry{ui}{
    type=\acronymtype, 
    name = {UI},
    plural = {UIs},
    description={User Interface}, 
    first={User Interface (UI)\glsadd{ui_g}}, 
    firstplural={User Interfaces (UIs)\glsadd{ui_g}},
    see=[Glossary:]{ui_g}
}

\newglossaryentry{gui_g}{
    name= {Graphical User Interface},
    plural={Graphical User Interfaces},
    description={A type of user interface that allows users to interact with a computer or software application through graphical elements such as icons, buttons, windows, and menus. Graphical User Interfaces make it easier for users to perform tasks and access functions by providing visual representations of actions and options. GUIs are commonly used in operating systems, software applications, and websites, enhancing user accessibility and usability.}
}

\newglossaryentry{gui}{
    type=\acronymtype, 
    name = {GUI},
    plural = {GUIs},
    description={Graphical User Interface}, 
    first={Graphical User Interface (GUI)\glsadd{gui_g}}, 
    firstplural={Graphical User Interfaces (GUIs)\glsadd{gui_g}},
    see=[Glossary:]{gui_g}
}

\newglossaryentry{ai_g}{
    name= {Artificial Intelligence},
    plural={Artificial Intelligences},
    description={A branch of computer science that focuses on creating intelligent machines capable of performing tasks that typically require human intelligence. Artificial Intelligence (AI) involves developing algorithms and models that enable computers to process data, learn from it, and make decisions or predictions. AI finds applications in a wide range of fields, including natural language processing, machine learning, robotics, and data analysis, revolutionizing industries and enhancing automation and problem-solving capabilities.}
}

\newglossaryentry{ai}{
    type=\acronymtype, 
    name = {AI},
    plural = {AI},
    description={Artificial Intelligence}, 
    first={Artificial Intelligence (AI)\glsadd{ai_g}}, 
    firstplural={Artificial Intelligences (AI)\glsadd{ai_g}},
    see=[Glossary:]{ai_g}
}

\newglossaryentry{rd_g}{
    name= {Research and Development},
    plural={Research and Developments},
    description={The process of systematically investigating, designing, and creating new products, technologies, or processes, as well as improving existing ones. Research and Development (R\&D) activities aim to enhance knowledge and capabilities, leading to innovation and advancements in various fields, including science, technology, and industry. R\&D plays a critical role in driving economic growth, competitiveness, and the development of cutting-edge solutions across sectors.}
}

\newglossaryentry{rd}{
    type=\acronymtype, 
    name = {R\&D},
    plural = {R\&D},
    description={Research and Development}, 
    first={Research and Development (R\&D)\glsadd{rd_g}}, 
    firstplural={Research and Developments (R\&D)\glsadd{rd_g}},
    see=[Glossary:]{rd_g}
}

\newglossaryentry{ctc_g}{
    name= {Circulating Tumor Cells},
    plural={Circulating Tumor Cells},
    description={Tumor cells that have detached from the primary tumor and entered the bloodstream or lymphatic system, allowing them to circulate throughout the body. Circulating Tumor Cells (CTCs) can provide valuable information about cancer progression, metastasis, and treatment effectiveness through a simple blood test. The detection and analysis of CTCs have implications for cancer diagnosis, treatment planning, and monitoring patient response to therapy.}
}

\newglossaryentry{ctc}{
    type=\acronymtype, 
    name = {CTC},
    plural = {CTCs},
    description={Circulating Tumor Cells}, 
    first={Circulating Tumor Cells (CTC)\glsadd{ctc_g}}, 
    firstplural={Circulating Tumor Cells (CTCs)\glsadd{ctc_g}},
    see=[Glossary:]{ctc_g}
}

\newglossaryentry{di_g}{
    name= {Digital Imaging},
    plural={Digital Imagings},
    description={The process of capturing, storing, and displaying visual information in digital format, allowing for the manipulation, analysis, and transmission of images using electronic devices and computer systems. Digital Imaging has transformed fields such as photography, medical imaging, and remote sensing, offering improved image quality, storage efficiency, and the ability to apply various image processing techniques. It plays a vital role in diverse applications, including healthcare, entertainment, and scientific research.}
}

\newglossaryentry{di}{
    type=\acronymtype, 
    name = {DI},
    plural = {DI},
    description={Digital Imaging}, 
    first={Digital Imaging (DI)\glsadd{di_g}}, 
    firstplural={Digital Imagings (DI)\glsadd{di_g}},
    see=[Glossary:]{di_g}
}

\newglossaryentry{shs_g}{
    name= {School of Health Sciences},
    plural={Schools of Health Sciences},
    description={An academic institution or department dedicated to the study and training of healthcare professionals in various disciplines, including medicine, nursing, pharmacy, and allied health fields. The School of Health Sciences (SHS) plays a crucial role in preparing students for careers in healthcare, research, and patient care, contributing to the advancement of medical knowledge and the improvement of healthcare services.}
}

\newglossaryentry{shs}{
    type=\acronymtype, 
    name = {SHS},
    plural = {SHS},
    description={School of Health Sciences}, 
    first={School of Health Sciences (SHS)\glsadd{shs_g}}, 
    firstplural={Schools of Health Sciences (SHS)\glsadd{shs_g}},
    see=[Glossary:]{shs_g}
}

\newglossaryentry{tesla_g}{
    name= {Tesla},
    plural={Teslas},
    description={A unit of measurement for the magnetic field strength used in Magnetic Resonance Imaging (MRI). The Tesla is the standard international unit, representing the strength of the magnetic field influencing the behavior of atomic nuclei during the imaging process. Higher Tesla values often contribute to improved image resolution and quality in MRI, impacting diagnostic capabilities in medical imaging.}
}

\newglossaryentry{tesla}{
    type=\acronymtype, 
    name = {T},
    plural = {T},
    description={Tesla}, 
    first={Tesla (T)\glsadd{tesla_g}}, 
    firstplural={Teslas (T)\glsadd{tesla_g}},
    see=[Glossary:]{tesla_g}
}

% \newglossaryentry{gauss_g}{
%     name= {Gauss},
%     description={A unit of measurement for the magnetic field strength, named after Carl Friedrich Gauss. In the context of magnetic fields, one Gauss is equal to one ten-thousandth of a Tesla (1 G = 10^-4 T). Gauss is commonly used in discussions related to magnetism and magnetic resonance, providing a scale for the intensity of magnetic fields.}
% }

% \newglossaryentry{gauss}{
%     type=\acronymtype, 
%     name = {G},
%     description={Gauss}, 
%     first={Gauss (G)\glsadd{gauss_g}}, 
%     see=[Glossary:]{gauss_g}
% }

\newglossaryentry{pathologist}{
    name= {Pathologist},
    plural={Pathologists},
    text= {pathologist},
    description={A medical professional who specializes in the study and diagnosis of diseases. Pathologists analyze tissues, cells, and bodily fluids to understand the nature and causes of illnesses. Their expertise is crucial in providing accurate diagnoses, guiding treatment decisions, and contributing to medical research.}
}

\newglossaryentry{gpr_g}{
    name= {Ground Penetrating Radar},
    plural={Ground Penetrating Radars},
    description={A geophysical method that uses radar pulses to image the subsurface. Ground Penetrating Radar (GPR) is employed for detecting and mapping features underground, such as utilities, archaeological artifacts, and geological structures. It plays a crucial role in various fields, including civil engineering, environmental assessment, and archaeological exploration.}
}

\newglossaryentry{gpr}{
    type=\acronymtype, 
    name = {GPR},
    plural = {GPR},
    description={Ground Penetrating Radar}, 
    first={Ground Penetrating Radar (GPR)\glsadd{gpr_g}}, 
    firstplural={Ground Penetrating Radars (GPR)\glsadd{gpr_g}},
    see=[Glossary:]{gpr_g}
}

\newglossaryentry{gpu_g}{
    name= {Graphics Processing Unit},
    plural={Graphics Processing Units},
    description={A specialized electronic circuit designed to accelerate the processing of images and videos for display on a computer screen. Graphics Processing Units (GPUs) are essential components in rendering realistic graphics for applications such as gaming, video editing, and complex simulations. Their parallel processing capabilities make them well-suited for handling large-scale graphical computations.}
}

\newglossaryentry{gpu}{
    type=\acronymtype, 
    name = {GPU},
    plural = {GPUs},
    description={Graphics Processing Unit}, 
    first={Graphics Processing Unit (GPU)\glsadd{gpu_g}}, 
    firstplural={Graphics Processing Units (GPUs)\glsadd{gpu_g}},
    see=[Glossary:]{gpu_g}
}

\newglossaryentry{fps_g}{
    name= {Frames Per Second},
    plural={Frames Per Second},
    description={A metric that measures the number of individual frames or images displayed in one second. In the context of digital displays, including augmented reality, a higher frame rate, measured in frames per second (FPS), contributes to smoother and more fluid visual experiences. Higher FPS is particularly important in applications such as gaming and video playback.}
}

\newglossaryentry{fps}{
    type=\acronymtype, 
    name = {FPS},
    plural = {FPS},
    description={Frames Per Second}, 
    first={Frames Per Second (FPS)\glsadd{fps_g}}, 
    firstplural={Frames Per Second (FPS)\glsadd{fps_g}},
    see=[Glossary:]{fps_g}
}

\newglossaryentry{sql_g}{
    name= {Structured Query Language},
    plural={Structured Query Languages},
    description={A domain-specific language used for managing and manipulating relational databases. Structured Query Language (SQL) provides a standardized way to interact with databases, enabling tasks such as data querying, insertion, updating, and deletion. SQL is widely employed in software development, data analysis, and database management systems.}
}

\newglossaryentry{sql}{
    type=\acronymtype, 
    name = {SQL},
    plural = {SQL},
    description={Structured Query Language}, 
    first={Structured Query Language (SQL)\glsadd{sql_g}}, 
    firstplural={Structured Query Languages (SQL)\glsadd{sql_g}},
    see=[Glossary:]{sql_g}
}

\newglossaryentry{gan_g}{
    name= {Generative Adversarial Nets},
    plural={Generative Adversarial Nets},
    description={A class of artificial intelligence algorithms introduced by Ian Goodfellow and his colleagues in 2014. Generative Adversarial Nets (GANs) consist of two neural networks, a generator, and a discriminator, which are trained simultaneously through adversarial training. GANs are widely used for generating realistic synthetic data, image-to-image translation, style transfer, and other tasks in the realm of artificial intelligence and machine learning.}
}

\newglossaryentry{gan}{
    type=\acronymtype, 
    name = {GAN},
    plural = {GANs},
    description={Generative Adversarial Nets}, 
    first={Generative Adversarial Nets (GAN)\glsadd{gan_g}}, 
    firstplural={Generative Adversarial Nets (GANs)\glsadd{gan_g}},
    see=[Glossary:]{gan_g}
}

\newglossaryentry{virt}{
    type=\acronymtype, 
    name = {VIRT},
    plural = {VIRTs},
    description={Volumetric Illustrative Rendering Techniques}, 
    first={Volumetric Illustrative Rendering Techniques (VIRT)}
}

\newglossaryentry{sdf_g}{
    name= {Sign Distance Field},
    plural={Sign Distance Fields},
    description={A mathematical representation used in computer graphics to define the distance from a point in space to the nearest surface of an object. Sign Distance Fields (SDFs) are commonly employed in rendering techniques such as ray marching and distance field rendering. They allow for efficient and accurate rendering of complex shapes and can be used in various applications, including augmented reality, computer-aided design, and video games.}
}

\newglossaryentry{sdf}{
    type=\acronymtype, 
    name = {SDF},
    plural = {SDFs},
    description={Sign Distance Field}, 
    first={Sign Distance Field (SDF)\glsadd{sdf_g}}, 
    firstplural={Sign Distance Fields (SDFs)\glsadd{sdf_g}},
    see=[Glossary:]{sdf_g}
}

\newglossaryentry{aabb_g}{
    name= {Axis Aligned Bounding Box},
    plural={Axis Aligned Bounding Boxes},
    description={A rectangular cuboid aligned with the coordinate axes, typically used in computer graphics and computational geometry to enclose a set of objects or to define the spatial extent of an entity. Axis Aligned Bounding Boxes (AABBs) are commonly employed in collision detection algorithms, rendering optimizations, and spatial partitioning techniques. They provide a simple and efficient way to approximate the geometry of complex shapes and facilitate various operations in virtual environments, including augmented reality applications.}
}

\newglossaryentry{aabb}{
    type=\acronymtype, 
    name = {AABB},
    plural = {AABBs},
    description={Axis Aligned Bounding Box}, 
    first={Axis Aligned Bounding Box (AABB)\glsadd{aabb_g}}, 
    firstplural={Axis Aligned Bounding Boxes (AABBs)\glsadd{aabb_g}},
    see=[Glossary:]{aabb_g}
}

\newglossaryentry{hci_g}{
    name= {Human-Computer Interaction},
    plural={Human-Computer Interactions},
    description={A field of study focusing on the design and use of computer technology, particularly the interactions between humans (the users) and computers. HCI is concerned with the ways humans interact with computers and design technologies that let humans interact with computers in novel ways. It involves the study of how people use technology, the design of user-friendly interfaces, and the development of new interaction techniques to improve the user experience.}
}

\newglossaryentry{hci}{
    type=\acronymtype, 
    name = {HCI},
    plural = {HCI},
    description={Human-Computer Interaction}, 
    first={Human-Computer Interaction (HCI)\glsadd{hci_g}}, 
    firstplural={Human-Computer Interactions (HCI)\glsadd{hci_g}},
    see=[Glossary:]{hci_g}
}

\newglossaryentry{jnd_g}{
    name= {Just Noticeable Difference},
    plural={Just Noticeable Differences},
    description={The smallest change in a stimulus that can be detected by an observer, typically defined as the threshold at which a difference becomes perceptible to a human sensory system. In the context of visual perception, the Just Noticeable Difference (JND) refers to the minimum change in brightness, color, or other visual attribute that can be perceived by the human eye. Understanding JND is crucial in various fields, including image processing, display technology, and user interface design, to ensure optimal user experiences and minimize perceptual errors.}
}

\newglossaryentry{jnd}{
    type=\acronymtype, 
    name = {JND},
    plural = {JND},
    description={Just Noticeable Difference}, 
    first={Just Noticeable Difference (JND)\glsadd{jnd_g}}, 
    firstplural={Just Noticeable Differences (JND)\glsadd{jnd_g}},
    see=[Glossary:]{jnd_g}
}

\newglossaryentry{pse_g}{
    name= {Point of Subject Equality},
    plural={Points of Subject Equality},
    description={In psychophysics, the Point of Subject Equality (PSE) refers to the stimulus intensity at which a subject perceives two stimuli as being equal in some specific aspect, such as brightness, loudness, or size. It is a key concept used to study perceptual phenomena and understand the relationship between physical stimuli and subjective perception. The determination of PSE plays a crucial role in various fields, including sensory research, human factors engineering, and user experience design.}
}

\newglossaryentry{slam_g}{
    name= {Simultaneous Localization and Mapping},
    plural={Simultaneous Localizations and Mappings},
    description={Simultaneous Localization and Mapping (SLAM) is a technique used in robotics and computer vision to construct a map of an unknown environment while simultaneously tracking the position of the observer within that environment. SLAM systems utilize sensor data, such as camera images or laser scans, to estimate the robot's trajectory and create a map of its surroundings. This technology is essential for autonomous navigation in various applications, including self-driving cars, drones, and mobile robots.}
}

\newglossaryentry{slam}{
    type=\acronymtype, 
    name = {SLAM},
    plural = {SLAM},
    description={Simultaneous Localization and Mapping}, 
    first={Simultaneous Localization and Mapping (SLAM)\glsadd{slam_g}}, 
    firstplural={Simultaneous Localizations and Mappings (SLAM)\glsadd{slam_g}},
    see=[Glossary:]{slam_g}
}

\newglossaryentry{pse}{
    type=\acronymtype, 
    name = {PSE},
    plural = {PSE},
    description={Point of Subject Equality}, 
    first={Point of Subject Equality (PSE)\glsadd{pse_g}}, 
    firstplural={Points of Subject Equality (PSE)\glsadd{pse_g}},
    see=[Glossary:]{pse_g}
}

\newglossaryentry{twofc_g}{
    name= {Two-alternative forced choice},
    plural={Two-alternative forced choices},
    description={Two-alternative forced choice (2AFC) is a psychophysical method used to measure the detectability or discriminability of a stimulus. In a 2AFC task, the observer is presented with two alternatives and must choose which one matches the target stimulus or is different from a reference stimulus. This method is commonly used in perceptual experiments to assess sensory thresholds and performance in various domains, such as vision, audition, and touch.}
}

\newglossaryentry{twofc}{
    type=\acronymtype, 
    name = {2AFC},
    plural = {2AFC},
    description={Two-alternative forced choice}, 
    first={Two-alternative forced choice (2AFC)\glsadd{twofc_g}}, 
    firstplural={Two-alternative forced choices (2AFC)\glsadd{twofc_g}},
    see=[Glossary:]{twofc_g}
}

\newglossaryentry{regression_analysis}{
    name= {Regression Analysis},
    plural={Regression Analyses},
    description={A statistical method used to examine the relationship between one dependent variable and one or more independent variables. The goal of regression analysis is to model the expected value of the dependent variable in terms of the independent variables, allowing for prediction and understanding of the underlying patterns. It is widely used in fields such as economics, biology, engineering, and social sciences.}
}


\newglossaryentry{classifier_g}{
    name= {Classifier},
    plural={Classifiers},
    description={A classifier is an algorithm in machine learning and statistics used to assign categories or labels to data points based on input features. It operates by learning patterns from labeled training data and then predicting the categories of new, unseen data. Classifiers are used in various applications such as image recognition, spam detection, medical diagnosis, and more.}
}

\newglossaryentry{non_euclidean_space}{
    name= {Non-Euclidean Space},
    plural={Non-Euclidean Spaces},
    description={A type of geometric space that is not based on the postulates of Euclidean geometry. In non-Euclidean space, the parallel postulate of Euclidean geometry does not hold, leading to the development of hyperbolic and elliptic geometries. These spaces have unique properties and are used in various fields, including physics, computer science, and cosmology, to model complex structures and phenomena.}
}

\newglossaryentry{perspective_corrected_projection}{
    name= {Perspective-Corrected Projection},
    plural={Perspective-Corrected Projections},
    description={A technique used in computer graphics to ensure that textures and objects are rendered accurately from the viewer's perspective. This method corrects distortions that can occur in standard projections, providing a more realistic and visually consistent representation of 3D objects on a 2D screen. Perspective-corrected projection is essential for applications such as virtual reality, gaming, and simulations to enhance the immersive experience and maintain visual fidelity.}
}

\newglossaryentry{exocentric_perception}{
    name= {Exocentric Perception},
    plural={Exocentric Perceptions},
    first={exocentric}, 
    description={A perspective in which an observer views the environment from an external viewpoint, as if looking at a scene from outside their own body. This contrasts with egocentric perception, where the observer's viewpoint is from within their own body. Exocentric perception is commonly used in virtual reality and other visualization technologies to provide a broader, more comprehensive understanding of spatial relationships and object interactions within an environment.}
}

\newglossaryentry{ordinal_perception}{
    name= {Ordinal Perception},
    plural={Ordinal Perceptions},
    first={ordinal},
    description={The ability to perceive and understand the relative order or ranking of objects or events without necessarily quantifying the differences between them. In visual perception, this often involves recognizing that one object is in front of another or that one event occurred before another. Ordinal perception is crucial for tasks that require an understanding of sequences and hierarchies.}
}

\newglossaryentry{bfs_g}{
    name= {Breadth-First Search},
    plural={Breadth-First Searches},
    description={An algorithm for traversing or searching tree or graph data structures. It starts at the root (or an arbitrary node) and explores all nodes at the present depth level before moving on to nodes at the next depth level. BFS is commonly used in shortest path problems in unweighted graphs, level-order traversal of a tree, and in scenarios where we need to explore all nodes at the same distance from the starting node.}
}

\newglossaryentry{bfs}{
    type=\acronymtype, 
    name = {BFS},
    plural = {BFS},
    description={Breadth-First Search}, 
    first={Breadth-First Search (BFS)\glsadd{bfs_g}}, 
    firstplural={Breadth-First Searches (BFS)\glsadd{bfs_g}},
    see=[Glossary:]{bfs_g}
}

\newglossaryentry{voxel_g}{
    name= {Voxel},
    plural={Voxels},
    text= {voxel},
    description={A volumetric pixel, or voxel, represents a value on a regular grid in three-dimensional space. Voxels are commonly used in 3D graphics, medical imaging, and scientific simulations to model spatial data, enabling efficient rendering and analysis of complex structures. Unlike traditional 2D pixels, voxels contain depth information, making them essential for volumetric rendering and spatial partitioning.}
}

\newcommand{\newglossaryentrywithacronym}[3]{
    %%% The glossary entry the acronym links to   
    \newglossaryentry{#1_gls}{
        name={#1},
        long={#2},
        description={#3}
    }

    % Acronym pointing to glossary
    \newglossaryentry{#1}{
        type=\acronymtype,
        name={#1},
        description={#2},
        first={#2 (#1)\glsadd{#1_gls}},
        see=[Glossary:]{#1_gls}
    }
}

