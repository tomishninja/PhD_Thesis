\chapter{Introduction}
This section here will cover the outlining issues regarding Augmented Reality in a medical setting. 

Indicate someone else's work and why this gap needs to be filled. Why there is a gap and what is the gap


\textbf{topics that fit well together}
\begin{itemize}
    \item explain at a very high level the goals of this thesis
    \item talk about what is MR
    \item talk about use cases of MR
    \item investigate the use cases of MR in medicine
    \item talk about what is X-ray vision
    \item talk at a high level about what X-ray vision could be used for (this should be expanded in Chapter 2)
\end{itemize}


\textbf{topics that don't fit as well together}
\begin{itemize}
    \item Volume rendering and polygonal rendering (Direct Volume Rendering and Indirect Volume Rendering)
    \item the link between artistic effects and X-ray vision (This may be better suited later on in this paper)
    \item talk about the hardware limitations
\end{itemize}

\section{Motivations}
At the very beginning of this PhD I was given the opportunity to travel to Siemens Forchheim, Bavaria, Germany as an intern in the department of SHS DI CT R&D CTC AI I was asked to create a system that was capable of overlaying the volumetric data that came out of the CT scanner and calibrating it to the bed so it could be overlaid over the patient. This proved to be a relatively simple task, one of the main issues I found was the data would always appear to be displayed in front of the patient from the perspective of the viewer. This can be seen in \textbf{TODO}


\section{Research Goals}
This section should address what this chapter is going to talk about 

\section{Research Questions}
TODO

\section{Approach}
This section should clearly state how this thesis is structured and what methods were used to achieve this.

\section{Contributions}
\begin{enumerate}
    \item A systematic literature review of X-ray vision
    \item A system that allows VST AR Effects to be displayed on an OST AR device
    \item A user study that compares various X-ray vision effects against each other.
    \item A perception-based user study looking at the benefit of x-ray vision effects for x-ray vision
    \item A 2FCA psychophysical study looking at the different individual tolerances to volume rendering and the effect that various x-ray vision effects can have on these effects. 
    \item A demonstration of Volumetric X-ray vision
\end{enumerate}

\section{Dissertation Structure}

\begin{enumerate}
    \item Introduction
    \item Literature review of X-ray vision
    \item Comparison of existing X-ray vision techniques
    \item Background of Volume Rendering
    \item A introduction to Volume Rendering
    \item The perception study
    \item The depth perception study
    \item future work
    \item Conclusion
\end{enumerate}
